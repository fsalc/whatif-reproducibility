%%%%%%%%%%%%%%%%%%%%%%%%%%%%%%%%%%%%%%%%
% Packages to use
%%%%%%%%%%%%%%%%%%%%%%%%%%%%%%%%%%%%%%%%
%\usepackage{epsfig}

\usepackage{amsmath}

\usepackage{graphicx}
\usepackage{balance}
\usepackage{bm}
\usepackage{color}
\usepackage{listings}
\usepackage{colortbl}
\usepackage{verbatim}
\usepackage{textcomp}
\usepackage{afterpage}
\usepackage{subcaption}

%\usepackage{amssymb}
\usepackage{amscd}
\usepackage{amsthm}
% \let\proof\relax
% \let\endproof\relax
\usepackage{mathtools}

\usepackage{algorithm}
\usepackage{algorithmicx}
\usepackage[noend]{algpseudocode}

\usepackage{tikz}

\usepackage{array}
\usepackage{tabularx}
\usepackage{multirow}
\usepackage{xspace}

\usetikzlibrary{shapes,backgrounds,arrows}
%\usepackage[textfont=bf,labelfont=bf,subrefformat=parens,position=top]{subfig}
\usepackage{listings}
\usepackage{caption}

% \usepackage[colorlinks=true,
%             urlcolor=blue,
%             citecolor=blue,
%             linkcolor=blue
%             ]{hyperref}
\usepackage{supertabular}
\usepackage{paralist}
\usepackage{cleveref}

\usepackage{enumitem}

%%%%%%%%%%%%%%%%%%%%%%%%%%%%%%%%%%%%%%%%
% Structure
%%%%%%%%%%%%%%%%%%%%%%%%%%%%%%%%%%%%%%%%
\newcommand{\partitle}[1]{\smallskip\noindent\textbf{#1}.}

%%%%%%%%%%%%%%%%%%%%%%%%%%%%%%%%%%%%%%%%
% Comments
%%%%%%%%%%%%%%%%%%%%%%%%%%%%%%%%%%%%%%%%
\DeclareRobustCommand{\BG}[1]{{\todo[color=red!40,inline]{\textbf{Boris says:}{#1}}}}
\DeclareRobustCommand{\FC}[1]{{\todo[color=red!40,inline]{\textbf{Felix says:}{#1}}}}
\DeclareRobustCommand{\BGDel}[2]{{\todo[inline,color=red!40]{\textbf{Boris deleted:}{#1}\textbf{because {#2}}}}}
\DeclareRobustCommand{\FCDel}[2]{{\todo[inline,color=red!40]{\textbf{Felix deleted:}{#1}\textbf{because {#2}}}}}
\DeclareRobustCommand{\BA}[1]{{\todo[inline,color=red!50]{\textbf{Bahareh says:}{#1}}}}
\DeclareRobustCommand{\DG}[1]{{\todo[inline,color=green!60]{\textbf{Dieter says:}{#1}}}}
\DeclareRobustCommand{\VR}[1]{{\todo[inline,color=blue!60]{\textbf{Venky says:}{#1}}}}
\DeclareRobustCommand{\sectionoverview}[1]{{\todo[inline,color=green!40]{\textbf{Section TODO and Overview}{#1}}}}

%%%%%%%%%%%%%%%%%%%%%%%%%%%%%%%%%%%%%%%%
% Figures
%%%%%%%%%%%%%%%%%%%%%%%%%%%%%%%%%%%%%%%%
\DeclareRobustCommand\fignumref[1]{\raisebox{.5pt}{\textcircled{\raisebox{-.75pt} {\small #1}}}}
\DeclareRobustCommand\rectangled[1]{%
      \tikz[baseline=(R.base)]\node[draw,rectangle,inner sep=0.5pt](R) {\small #1};\!
    }

%%%%%%%%%%%%%%%%%%%%%%%%%%%%%%%%%%%%%%%%
% Algorithms
%%%%%%%%%%%%%%%%%%%%%%%%%%%%%%%%%%%%%%%%
\newcommand{\algtab}{\thickspace\thickspace\thickspace\thickspace\thickspace\thickspace}
\renewcommand*\Call[2]{\textproc{\textcolor{darkdarkpurple}{#1}}(#2)}
\newcommand{\card}[1]{\left|{#1}\right|}


%%%%%%%%%%%%%%%%%%%%%%%%%%%%%%%%%%%%%%%%
% Theorems, Definitions, Examples
%%%%%%%%%%%%%%%%%%%%%%%%%%%%%%%%%%%%%%%%
\newtheorem{theo}{Theorem}
\newtheorem{lem}[theo]{Lemma}
\newtheorem{propo}[theo]{Proposition}
\newtheorem{coll}[theo]{Corollary}
\newtheorem{exam}{Example}
\newtheorem{defi}{Definition}
%\newenvironment{proofsketch}{\paragraph{Proofsketch:}}{\hfill}
\newenvironment{theoremproof}[2][\bf Theorem]{\begin{trivlist}
\item[\hskip \labelsep {#1}\hskip \labelsep {#2}] \em}{\end{trivlist}}
\newcommand{\proofpar}[1]{\smallskip\noindent\underline{{#1}}}

%%%%%%%%%%%%%%%%%%%%%%%%%%%%%%%%%%%%%%%%
% Colors
%%%%%%%%%%%%%%%%%%%%%%%%%%%%%%%%%%%%%%%%
\definecolor{black}{rgb}{0,0,0}
\definecolor{grey}{rgb}{0.8,0.8,0.8}
\definecolor{red}{rgb}{1,0,0}
\definecolor{green}{rgb}{0,1,0}
\definecolor{darkgreen}{rgb}{0,0.5,0}
\definecolor{darkpurple}{rgb}{0.5,0,0.5}
\definecolor{darkdarkpurple}{rgb}{0.3,0,0.3}
\definecolor{blue}{rgb}{0,0,1}
\definecolor{shadegreen}{rgb}{0.95,1,0.95}
\definecolor{shadeblue}{rgb}{0.95,0.95,1}
\definecolor{shadered}{rgb}{1,0.85,0.85}
\definecolor{shadegrey}{rgb}{0.85,0.85,0.85}
\definecolor{oddRowGrey}{rgb}{0.80,0.80,0.80}
\definecolor{evenRowGrey}{rgb}{0.85,0.85,0.85}

%%%%%%%%%%%%%%%%%%%%%%%%%%%%%%%%%%%%%%%%
% Math utils
%%%%%%%%%%%%%%%%%%%%%%%%%%%%%%%%%%%%%%%%
\DeclareMathOperator*{\argmax}{arg\,max}
\DeclareMathOperator*{\argmin}{arg\,min}
\newcommand{\concat}{\triangleright}
\newcommand{\defas}{:=}
\newcommand{\mathtext}[1]{\thickspace\text{#1}\thickspace}
\newcommand{\mathtab}{\thickspace\thickspace\thickspace}

%%%%%%%%%%%%%%%%%%%%%%%%%%%%%%%%%%%%%%%%
% Relational Algebra
%%%%%%%%%%%%%%%%%%%%%%%%%%%%%%%%%%%%%%%%
\newcommand{\projection}{\Pi}
\newcommand{\selection}{\sigma}
\newcommand{\aggregation}{\gamma}
\newcommand{\union}{\cup}
\newcommand{\intersection}{\cap}
\newcommand{\difference}{-}
\newcommand{\rename}{\rho}

\def\ojoin{\setbox0=\hbox{$\bowtie$}%
  \rule[-.02ex]{.25em}{.4pt}\llap{\rule[\ht0]{.25em}{.4pt}}}
\def\leftouterjoin{\mathbin{\ojoin\mkern-5.8mu\bowtie}}
\def\rightouterjoin{\mathbin{\bowtie\mkern-5.8mu\ojoin}}
\def\fullouterjoin{\mathbin{\ojoin\mkern-5.8mu\bowtie\mkern-5.8mu\ojoin}}

\newcommand{\join}{\bowtie}
\newcommand{\crossprod}{\times}
\newcommand{\duprem}{\delta}
\newcommand{\win}{\omega}
\newcommand{\asingleton}[2]{\{#1 \to {#2}\}}
%\newcommand{\asingleton}[2]{\{#1\}}

\newcommand{\denserank}{\#}
\newcommand{\rownum}{\filledmedtriangledown}
\newcommand{\firstf}{\textsc{first}}

\newcommand{\schema}[1]{\textsc{Sch}(#1)}
\newcommand{\rschema}[1]{\schema{#1}}
\newcommand{\schemaV}[2]{\textsc{Sch}^{#1}(#2)}
\newcommand{\nullList}[1]{\textsc{Null}(#1)}

\newcommand{\T}{\mathbf{true}}
\newcommand{\F}{\mathbf{false}}
\newcommand{\isnull}{\,\mathbf{isnull}}

\newcommand{\sqlCase}[3]{\mathbf{if}\thickspace #1 \thickspace
  \mathbf{then} \thickspace #2 \thickspace \mathbf{else} \thickspace #3}

\newcommand{\RAPlus}{{\cal RA}^+}
\newcommand{\RAPlusAnn}{{\cal RA}^{+/\genAnnotOp}}
\newcommand{\RAPlusK}[1]{\RAPlus_{#1}}

%%%%%%%%%%%%%%%%%%%%%%%%%%%%%%%%%%%%%%%%
% Query Containment and Equivalence
%%%%%%%%%%%%%%%%%%%%%%%%%%%%%%%%%%%%%%%%
\newcommand{\qContain}{{\sqsubseteq}}

%%%%%%%%%%%%%%%%%%%%%%%%%%%%%%%%%%%%%%%%
% Rewrite Rules
%%%%%%%%%%%%%%%%%%%%%%%%%%%%%%%%%%%%%%%%

\newcommand{\nulleq}{=_{\epsilon}}
\newcommand{\PName}{{\cal P}}
\newcommand{\annAttrs}{{\cal P}}
\newcommand{\upAttr}{{\cal U}}
\newcommand{\tidAttr}{{Id}}
\newcommand{\versionAttr}{V}
\newcommand{\xidAttr}{Xid}
\newcommand{\relRewr}{{\textsc{Rew}}}

%%%%%%%%%%%%%%%%%%%%%%%%%%%%%%%%%%%%%%%%
% Transactions
%%%%%%%%%%%%%%%%%%%%%%%%%%%%%%%%%%%%%%%%
\newcommand{\xid}{T}
\newcommand{\xidDomain}{\mathbb{T}}
\newcommand{\version}{\nu}
\newcommand{\versionDomain}{\mathbb{V}}
\newcommand{\tid}{{id}}
\newcommand{\tidOf}{\tid}
\newcommand{\tidDomain}{\mathbb{I}}
\newcommand{\start}[1]{Start({#1})}
\newcommand{\finish}[1]{End(#1)}
\newcommand{\hdb}{\db_{\history}}
\newcommand{\curDB}[2]{#1_{#2}^{Cur}}

\newcommand{\maxQSize}{\textsc{Max}}
\newcommand{\numUpInHist}{\#U}

\newcommand{\lastUp}{\textsc{Last}}

\newcommand{\ccMechanism}{CC}

%%%%%%%%%%%%%%%%%%%%%%%%%%%%%%%%%%%%%%%%
% Update operations
%%%%%%%%%%%%%%%%%%%%%%%%%%%%%%%%%%%%%%%%
\newcommand{\update}[2]{\mathcal{U}_{#1,#2}}
\newcommand{\aupdate}{\update{\pset}{\cond}}
\newcommand{\delete}[1]{\mathcal{D}_{#1}}
\newcommand{\adelete}{\delete{\cond}}
\newcommand{\ins}[1]{\mathcal{I}_{#1}}
\newcommand{\ainsert}{\ins{t}}
\newcommand{\aqinsert}{\ins{\query}}
\newcommand{\up}{u}
\newcommand{\upPos}{pos}
\newcommand{\notup}{\bottom}
\newcommand{\pos}{\mathsf{Pos}}

\newcommand{\relUpdate}[5]{{\cal U}[#1,#2,#3,#4](#5)}
\newcommand{\aRelUpdate}{{\cal U}}
\newcommand{\relInsert}[4]{{\cal I}[#1,#2,#3](#4)}
\newcommand{\aRelInsert}{{\cal I}}
\newcommand{\relDelete}[4]{{\cal D}[#1,#2,#3](#4)}
\newcommand{\aRelDelete}{{\cal D}}
\newcommand{\commitOp}[3]{{\cal C}[#1,#2](#3)}
\newcommand{\aDoCommit}{\textsc{com}}
\newcommand{\doCommit}[3]{\aDoCommit[#1,#2](k)}

%%%%%%%%%%%%%%%%%%%%%%%%%%%%%%%%%%%%%%%%
% Versions and Provenance
%%%%%%%%%%%%%%%%%%%%%%%%%%%%%%%%%%%%%%%%
\newcommand{\db}{D}
\newcommand{\tup}{t}
\newcommand{\rel}{R}
\newcommand{\relSchema}{\mathbf{\rel}}
\newcommand{\dbSchema}{\mathbf{\db}}
\newcommand{\dataDomain}{\mathbb{D}}
\newcommand{\relsOf}[1]{\textsc{rels}(#1)}

% \newcommand{\upMarker}[4]{{}_{#1}^{#2}\tau_{#3}^{#4}}
%\newcommand{\upMarker}[4]{\tau_{[#2,#3,#1]}}
\newcommand{\upMarker}[4]{#1_{#2,#3}^{#4}}
\newcommand{\cMarker}[3]{C_{#1,#2}^{#3}}
\newcommand{\genUpMarker}{{\cal A}}
%\newcommand{\upMarker}[4]{#1_{#2}^{#3}}
%\newcommand{\upMarker}[4]{\tau_{#1,#2,#3}^{#4}}
\newcommand{\upMarkers}{\mathbb{A}}
\newcommand{\upMark}{{\cal A}}
\newcommand{\insMark}{I}
\newcommand{\delMark}{D}
\newcommand{\updMark}{U}
\newcommand{\withEnd}{\textsc{endV}}

\newcommand{\relFV}[2]{{#1}[#2]}
\newcommand{\relV}[3]{{#1}[#2,#3]}
\newcommand{\relCV}[2]{{#1}[#2]}
\newcommand{\relSSI}[3]{{#1}_{#2}[#3]}
\newcommand{\relVRange}[3]{{#1}[#2,#3]}
\newcommand{\dbV}[2]{{\db}[#1,#2]}
\newcommand{\dbver}[1]{\db_{#1}}

\newcommand{\snapshot}[2]{{#1}_{#2}}
% \newcommand{\timeSlice}[3]{#1_{[#2,#3]}}

\newcommand{\genAnnotOp}{\alpha}
\newcommand{\annotOp}[4]{\genAnnotOp_{{#1},{#2},{#3}}}

\newcommand{\vFilt}[1]{{\gamma}_{#1}}
\newcommand{\nullV}{\emptyset}
\newcommand{\hUnv}{h_U}
\newcommand{\unversion}{\textsc{Unv}}

\newcommand{\validAt}{\textsc{validAt}}
\newcommand{\hasUp}{\textsc{updated}}

\newcommand{\partP}[1]{\db[#1]}
\newcommand{\provTFilt}[1]{filt(#1)}
\newcommand{\provAFilt}{h_{f}}

\newcommand{\relEnc}[1]{\textsc{Rel}(#1)}
\newcommand{\relEncI}[2]{\textsc{Rel}^{#1}(#2)}
\newcommand{\upEnc}[2]{\textsc{EncU}^{#1}(#2)}
\newcommand{\rEnc}[1]{\textsc{EncR}(#1)}
\newcommand{\idGen}{ID_{\cal P}}
\newcommand{\idGenUse}[1]{ID_{#1}}
%%%%%%%%%%%%%%%%%%%%%%%%%%%%%%%%%%%%%%%%
% Semirings
%%%%%%%%%%%%%%%%%%%%%%%%%%%%%%%%%%%%%%%%
\newcommand{\boolSR}{\mathbb{B}}
\newcommand{\boolSRV}{\mathbb{B}^{\version}}
\newcommand{\bagSR}{\mathbb{N}}
\newcommand{\bagSRV}{\mathbb{N}^{\version}}

\newcommand{\ppSR}{\mathbb{N}[X]}
\newcommand{\ppSRV}{\mathbb{N}[X]^{\version}}

%%%%%%%%%%%%%%%%%%%%%%%%%%%%%%%%%%%%%%%%
% Reenactment queries
%%%%%%%%%%%%%%%%%%%%%%%%%%%%%%%%%%%%%%%%
\newcommand{\ractSymbol}{\mathcal{R}}
\newcommand{\ract}[1]{\ractSymbol_{#1}}
\newcommand{\subract}[2]{\ractSymbol^{#1}_{#2}}
\newcommand{\subractR}[1]{\ractSymbol^{#1}}
\newcommand{\reduce}{\rightsquigarrow}

%%%%%%%%%%%%%%%%%%%%%%%%%%%%%%%%%%%%%%%%
% Semirings and Multiversion Semiring
%%%%%%%%%%%%%%%%%%%%%%%%%%%%%%%%%%%%%%%%
\newcommand{\semK}{{\cal K}}
\newcommand{\mvK}{{{\cal K}^{\version}}}
\newcommand{\mvOf}[1]{{#1}^{\version}}
\newcommand{\mvDom}{K^{\version}}
\newcommand{\NX}{\mathbb{N}[X]}
\newcommand{\NXv}{\mathbb{N}[X]^{\version}}
\newcommand{\semN}{\mathbb{N}}
\newcommand{\congr}[1]{[{#1}]_\sim}
\newcommand{\congrEq}{\equiv_\sim}
\newcommand{\numInSum}[1]{n(#1)}
\newcommand{\nthOfK}[2]{#1[#2]}
\newcommand{\mul}{\cdot}

\newcommand{\liftH}[1]{{#1}^{\version}}
\newcommand{\SReval}{Eval_{\varAssign}}
\newcommand{\aSReval}[1]{Eval_{#1}}
\newcommand{\SRVeval}{{Eval_{\varAssign}}^\version}
\newcommand{\aSRVeval}[1]{{\SReval{#1}}^\version}

%%%%%%%%%%%%%%%%%%%%%%%%%%%%%%%%%%%%%%%%
% What-if
%%%%%%%%%%%%%%%%%%%%%%%%%%%%%%%%%%%%%%%%
\newcommand{\query}{Q}
\newcommand{\whatif}{\mathcal{W}}
\newcommand{\deltaDB}{\Delta \db}
\newcommand{\deltaResult}{\Delta \query}

\newcommand{\hwhatif}{\mathcal{H}}
\newcommand{\ahwhatif}{(\history,\db,\deltaHist)}
\newcommand{\history}{H}
\newcommand{\histpre}[1]{\history_{#1}}
\newcommand{\hpreOf}[2]{{#1}_{#2}}
\newcommand{\histslice}[2]{\history_{#1,#2}}
\newcommand{\hsliceOf}[3]{{#1}_{#2,#3}}
\newcommand{\idxs}{\mathcal{I}}
\newcommand{\idx}{i}
\newcommand{\hislice}[1]{\history_{#1}}
\newcommand{\hisliceOf}[2]{{#1}_{#2}}
\newcommand{\hOrder}{\leq_{\history}}

% MODIFICATIONS
\newcommand{\modi}{m}
\newcommand{\mdel}[1]{\ensuremath{\mathbf{del}(#1)}\xspace}
\newcommand{\minsert}[2]{\ensuremath{\mathbf{ins}_{#2}(#1)}\xspace}

\newcommand{\hSorder}{<_{\history}}
\newcommand{\deltaHist}{\mathcal{M}}
\newcommand{\hmod}[2]{#1[#2]}
\newcommand{\ahmod}{\hmod{\history}{\deltaHist}}
\newcommand{\hdeltaResult}{\Delta \hwhatif}
\newcommand{\qResultDiff}[3]{\Delta(#1,#2,#3)}
\newcommand{\diffsym}{\Delta}
\newcommand{\iDiff}[2]{\diffsym(#1,#2)}
\newcommand{\hTsubst}[2]{\history[{#1} \gets {#2}]}
\newcommand{\applyM}[2]{#1[#2]}
\newcommand{\hSubst}[3]{\history[{#1}.{#2} \gets {#3}]}
\newcommand{\commit}{c}
\newcommand{\symExe}[2]{\history[#1](S[#2])}
\newcommand{\symCond}[2]{\cond(\history[#1](S[#2]))}
\newcommand{\symIns}[1]{S[#1]}
\newcommand{\pset}{\textit{Set}}
\newcommand{\psetOf}[1]{\pset_{#1}}
\newcommand{\attr}[1]{A_{#1}}
\newcommand{\setAttr}[3]{\pset_{\history[#1]}(\symIns{#2}.\attr{#3})}
\newcommand{\bVar}[2]{x_{\history[#1],\symIns{#2}}}
\newcommand{\cond}{\theta}
\newcommand{\condOf}[1]{\theta_{#1}}
\newcommand{\expr}{e}
\newcommand{\subst}[3]{{#1}[#2 \gets #3]}
\newcommand{\cons}{c}
\newcommand{\condbf}{\boldsymbol{\phi}}
\newcommand{\exprbf}{\bf{e}}
\newcommand{\consbf}{\bf{c}}
\newcommand{\varbf}{\bf{v}}
\newcommand{\depUp}{\mathsf{Dep}}
\newcommand{\depOver}{\depUp^{\ast}}
\newcommand{\hslice}[2]{{#1}_{#2}}
\newcommand{\aSlice}{\hslice{\history}{\idxs}}
\newcommand{\removeUp}[2]{\mathsf{Del}({#1}, {#2})}

\newcommand{\exclusion}[3]{\zeta\left({#1}, {#2}, {#3}\right)}
\newcommand{\slicetest}[3]{\zeta({#1},{#2},{#3})}
\newcommand{\aslicetest}{\slicetest{\hwhatif}{\idxs}{\adbconstr}}


%%%%%%%%%%%%%%%%%%%%%%%%%%%%%%%%%%%%%%%%%%%%%%%%%%%%%%%%%%%%%%%%%%%%%%%%%%%%%%%%
% DS
%%%%%%%%%%%%%%%%%%%%%%%%%%%%%%%%%%%%%%%%%%%%%%%%%%%%%%%%%%%%%%%%%%%%%%%%%%%%%%%%
\newcommand{\qDSh}{\query_{\history}^{DS}}
\newcommand{\qDSm}{\query_{\history[\deltaHist]}^{DS}}
\newcommand{\condDS}[2]{\cond_{#1}^{DS}(#2)}
\newcommand{\condDSh}[1]{\condDS{\history}{#1}}
\newcommand{\condDSm}[1]{\condDS{\history[\deltaHist]}{#1}}
\newcommand{\acondDS}{\condDS{\history}{\up}}
\newcommand{\pushCond}[2]{{#1}\downarrow^{#2}}
\newcommand{\qpushCond}[2]{(#1)\downarrow^{#2}}

\newcommand{\mcondDS}[3]{\cond_{#1}^{DS}[#3](#2)}
\newcommand{\mcondDSh}[2]{\mcondDS{\history}{#1}{#2}}
\newcommand{\mcondDSm}[2]{\condDS{\history[\deltaHist]}{#1}{#2}}
\newcommand{\amcondDS}{\condDS{\history}{\up}{\rel}}
\newcommand{\mpushCond}[3]{{#1}[#3]\downarrow^{#2}}
\newcommand{\mqpushCond}[3]{(#1)[#3]\downarrow^{#2}}




%%%%%%%%%%%%%%
%%% Incomplete DBS %%%
%%%%%%%%%%%%%%
\newcommand{\world}{\db}
\newcommand{\worlds}{\mathcal{D}}


%%%%%%%%%%%%%%
%%% VC DBs %%%
%%%%%%%%%%%%%%
\newcommand{\varDom}{\Sigma}
\newcommand{\varAssign}{\lambda}
\newcommand{\allVarAssigns}{\Lambda}
\newcommand{\vcdb}{\mathbf{D}}
\newcommand{\vcrel}{\mathbf{R}}
\newcommand{\vct}{\mathbf{t}}
\newcommand{\singtupH}[1]{\vct_{#1}}
\newcommand{\vcOf}[1]{\mathbf{#1}}
\newcommand{\vctn}{\vcOf{t_{new}}}
\newcommand{\vcdbini}{\vcOf{\db_0}}
\newcommand{\vcdbver}[1]{\vcOf{\db_{#1}}}
\newcommand{\lcond}{\phi}
\newcommand{\gcond}{\Phi}
\newcommand{\dbconstr}[1]{\gcond_{#1}}
\newcommand{\adbconstr}{\dbconstr{\db}}
\newcommand{\worldsOf}[1]{Mod({#1})}


%%%%%%%%%%%%%%%%%%%%%%%%%%%%%%%%%%%%%%%%
% MILP translation
%%%%%%%%%%%%%%%%%%%%%%%%%%%%%%%%%%%%%%%%
\DeclareRobustCommand{\compRule}[2]{\frac{{#1}{#2}}}
\newcommand{\upBound}{M}
\newcommand{\var}{v}
\newcommand{\bvar}{b}

%%%%%%%%%%%%%%%%%%%%%%%%%%%%%%%%%%%%%%%%
% Tables
%%%%%%%%%%%%%%%%%%%%%%%%%%%%%%%%%%%%%%%%
\newcommand{\thead}[1]{{\cellcolor{black}{\textcolor{white}{\textbf{#1}}}}}
\newcommand{\lthead}[1]{{\cellcolor{evenRowGrey}{\textbf{#1}}}}
\newcommand{\evenRow}{\rowcolor{evenRowGrey}}
\newcommand{\oddRow}{\rowcolor{oddRowGrey}}
\newcommand{\hlrow}{\rowcolor{shadered}}

%%%%%%%%%%%%%%%%%%%%%%%%%%%%%%%%%%%%%%%%
% Examples
%%%%%%%%%%%%%%%%%%%%%%%%%%%%%%%%%%%%%%%%
\newcommand{\orderid}[1]{oid_{#1}}

%%%%%%%%%%%%%%%%%%%%%%%%%%%%%%%%%%%%%%%%
% Subfloat that works with verbatim
%%%%%%%%%%%%%%%%%%%%%%%%%%%%%%%%%%%%%%%%
    \makeatletter
     \newbox\sf@box
     \newenvironment{SubFloat}[2][]%
       {\def\sf@one{#1}%
        \def\sf@two{#2}%
        \setbox\sf@box\hbox
          \bgroup}%
       {  \egroup
        \ifx\@empty\sf@two\@empty\relax
          \def\sf@two{\@empty}
        \fi
        \ifx\@empty\sf@one\@empty\relax
          \subfloat[\sf@two]{\box\sf@box}%
        \else
          \subfloat[\sf@one][\sf@two]{\box\sf@box}%
        \fi}
     \makeatother

\newcommand{\eat}[1]{}
\hyphenation{prov-e-nance}

\newenvironment{proofsketch}% environment name
{% begin code
\noindent \textsc{Proof Sketch.}%
}%
{\qedsymbol}% end code

%%%%%%%%%%%%%%%%%%%%%%%%%%%%%%%%%%%%%%%%
% abbreviations
%%%%%%%%%%%%%%%%%%%%%%%%%%%%%%%%%%%%%%%%
\newcommand{\abbrHW}{HWQ\xspace}
\newcommand{\abbrHWs}{HWQs\xspace}
\newcommand{\termHW}{historical what-if query\xspace}

%%%%%%%%%%%%%%%%%%%%%%%%%%%%%%%%%%%%%%%%
% main document for emacs auctex
%%%%%%%%%%%%%%%%%%%%%%%%%%%%%%%%%%%%%%%%



%%% Local Variables:
%%% mode: latex
%%% TeX-master: "historical_whatif"
%%% End:
