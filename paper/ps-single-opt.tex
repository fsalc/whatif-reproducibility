\section{Optimized Program Slicing for Single Modifications}
\label{sec:optim-progr-slic}

Based on the VC-databases created by this step we then determine a static slice. We introduce a condition called \textit{dependency} that can be checked over the VC-database and determines whether an update's result depends on the modification $\deltaHist$. As we will demonstrate, the set of dependent updates from $\history$ ($\history(\deltaHist)$) is a static slice for the historical what-if query $\hwhatif$.

Observe that a statement can be excluded from reenactment if none of the tuples affected by the statement will be in the difference between $\history$ and $\history[\deltaHist]$. Note that any tuple in the difference has to be affected by at least one of the statements modified by a modification $\modi \in \deltaHist$, because a tuple that is not affected by any statement from $\deltaHist$ will be the same in $\history$ and $\history[\deltaHist]$ and, thus, cannot be in the result of $\hwhatif$.

Conversely, an update has to be included in a static slice if there exists at least one database $\db$ that contains a tuple which is affected by the statement and which is in the result of $\hwhatif$ over $\db$.

%
%
%

%%%%%%%%%%%%%%%%%%%%%%%%%%%%%%%%%%%%%%%%
\begin{defi}
Let $\history$ be a history and $\history[\deltaHist]$ be a modified history, $t$ be a tuple in the relation, and $u_{old}$ be an unmodified update and $u_{new}$ the corresponding modified update for any modification $m \in \deltaHist$. Let $t_i$ be the tuple at $\history_i(t)$ ($\ahmod_i(t)$). Let $\pos(u)$ be the position of update $u$ in $\history$. We define the following condition for exclusion from a non-minimal slice of the history:
\begin{align*}
\exclusion{H}{\deltaHist}{u_i} =\ &\forall m \in \deltaHist \neg\exists \varAssign \exists t \in \db_{\history[\deltaHist], \pos(u_m)}\\
&(\cond_{u_{orig}}(\varAssign(t_{\pos({u_{m}})})) \land \cond_{u_{i}}(\varAssign(t_{i-1})))\\
&\lor (\cond_{u_{new}}(\varAssign(t_{\pos({u_{m}})})) \land \cond_{u_{i}}(\varAssign(t_{i-1})))
\end{align*}
\end{defi}

%%%%%%%%%%%%%%%%%%%%%%%%%%%%%%%%%%%%%%%%

We use VC-Tables for symbolic execution of update operations and determining dependency of updates on  modifications by the historical what-if queries. Independent updates can be excluded from reenactment as their output is the same in $\history$ and $\history[\deltaHist]$.


%%%%%%%%%%%%%%%%%%%%%%%%%%%%%%%%%%%%%%%%
%
%


We can determine dependent updates in the history by generating VC-tables for each $u$ and $ u'$ whereas $m = u \gets u'$ and $m \in \deltaHist$. Then, we apply symbolic execution on these VC-tables for the remaining updates in the history. For each examined update ($u_i$), if we can generate a possible word for a tuple that is modified either by both $u_i$ and $u$ or $u_i$ and $u'$, $u_i$ is a dependent update. Since, a possible world shows there is a possibility that a tuple was modified by the historical what-if query and the examined update which must be considered in the answer of the the historical what-if query.




%%%%%%%%%%%%%%%%%%%%%%%%%%%%%%%%%%%%%%%%
\begin{exam}
In order to detect dependent update in \Cref{fig:running-vctb}, we examine generating a possible world for a tuple in the VC-Table that is modified by both the first ($u_1$) and the second update ($u_2$) in the history in \Cref{fig:Transitive-Transactions-Example}.
 After symbolic execution of the second update where $\cond:= x_{Country}=UK \wedge x_{Price} <=100$, there are four tuples in the VC-Table. The first tuple which has the conditional function
 $x_{Price} >=40 \wedge (x_{Country}=UK \wedge x_{Price} <=100)$ representing it is modified by both updates. The possible world can be generated by evaluating $(x_{Country} \leftarrow UK,x_{Price} \leftarrow 40,x_{ShippingFee} \leftarrow 5)(x_{Price}>=40 \wedge (x_{Country}=UK \wedge x_{Price} <=100))$. As $40>=40 \wedge (UK=UK \wedge 40 <=100):= true$. The possible world $\world''$ after executing the first and second update statements can be $\world''=\lbrace UK,40,5\rbrace$. So, the second update is dependent on the first update as it is possible that  a tuple is modified by both updates.
\end{exam}
\FC{Update example to use global conditions?}
%%%%%%%%%%%%%%%%%%%%%%%%%%%%%%%%%%%%%%%%




%%%%%%%%%%%%%%%%%%%%%%%%%%%%%%%%%%%%%%%
\begin{theo}
%
Consider a historical what-if query $\hwhatif = \ahwhatif$ with a single modification $\deltaHist = \{ m\}$ for $m = \up_1 \gets \up_1'$ over history $\history= u_1,\ldots,u_{n}$.
%
%
The set $\idxs = \{ i \mid \neg \exclusion{\history}{\deltaHist}{u_i} \}$ is a slice for $\hwhatif$.
%
%
%
%
%
%
%
%
%
%
\end{theo}
%%%%%%%%%%%%%%%%%%%%%%%%%%%%%%%%%%%%%%%%
\ifnottechreport{
  \begin{proofsketch}
    We prove this theorem by contradiction, showing that it is not possible for $\zeta$ to be true over a dependent update. For the full proof see \cite{techreport}.
  \end{proofsketch}
}
\iftechreport{
  \begin{proof}

\newcommand{\iex}{\idxs_{excl}}
\newcommand{\iin}{\idxs_{in}}
\newcommand{\iexi}[1]{\idxs_{excl,#1}}
\newcommand{\iini}[1]{\idxs_{in,#1}}
\newcommand{\dslicei}[1]{\Delta_{#1}}

    % 
  Consider a history $\history = (u_1, \ldots, u_n)$, set of modifications $\deltaHist = \{m\}$ for $m = u_1 \gets u_1'$, and historical what-if query $\hwhatif = (\history, \db, \deltaHist)$. Let $\idxs_{in} = \{ \upPos(u_i) \mid \neg \exclusion{\history}{\deltaHist}{u_i} \}$ and $\iex = \{  \upPos(u_i) \mid \exclusion{\history}{\deltaHist}{u_i} \}$. We have to prove that $\idxs_{in}$ is a slice for $\hwhatif$. In the following let $\iexi{i}$ denote the first $i$ positions from $\iex$ and $\iini{i} = [1,n] - \iexi{i}$, i.e., from the histories all updates from $\iexi{i}$. We prove the theorem by induction over $i$. In the following we use $\Delta$ to denote $\iDiff{\history(\db)}{\ahmod(\db)}$ and $\dslicei{i}$ to denote $\iDiff{\hslice{\history}{\iini{i}}(\db)}{\hslice{\ahmod}{\iini{i}}(\db)}$.

%%%%%%%%%%%%%%%%%%%%%%%%%%%%%%%%%%%%%%%%%%%%%%%%%%%%%%%%%%%%%%%%%%%%%%%%%%%%%%%%
\proofpar{Base case:}
Consider $\iexi{1} = {j}$ for some $j \in [1,n]$. To prove that $\iini{1}$ is a slice, we have to show that $\Delta = \dslicei{1}$. For sake of contradiction assume that $\Delta \neq \dslicei{1}$. Then there has to exist a tuple $\tup$ such that (i) $\tup \in \Delta \land \tup \not\in \dslicei{1}$ or (ii) $\tup \not\in \Delta \land \tup \in \dslicei{1}$.

%%%%%%%%%%%%%%%%%%%%%%%%%%%%%%%%%%%%%%%%
\proofpar{$\tup \in \Delta \land \tup \not\in \dslicei{1}$:}
If $\tup \in \Delta$ then either $\tup \in \history(\db) \land \tup \not\in \ahmod(\db)$ or $\tup \not\in \history(\db) \land \tup \in \ahmod(\db)$. Since these two cases are symmetric, WLOG assume that $\tup \in \history(\db) \land \tup \not\in \ahmod(\db)$. It follows that $\exists \tup_{1} \in \db$ such that $\history(\tup_1) = \tup$ and $\neg \exists \tup_{1}' \in \db$ such that $\ahmod(\tup_{1}') = \tup$. Specifically, $\ahmod(\tup_{1}') \neq \tup$. Now let $\tup_i = \hslice{\history}{i}(\tup_1)$ and $\tup_i'= \hslice{\ahmod}{i}(\tup_1)$. Since $j \in \iex$, condition $\exclusion{\history}{\deltaHist}{u_j}$ has to hold which implies $\neg \theta_j(\tup_j)$ and $\neg \theta_j(\tup_{j}')$.

Now let us use $s_i$ to denote $\hslice{\history}{\iini{1} \cap [1,i]}(\tup_1)$ and $s_i'$ to denote $\hslice{\ahmod}{\iini{1} \cap [1,i]}(\tup_1)$.
Since $\hslice{\history}{j-1} = \hslice{\history}{\iini{1} \cap [1,j-1]}$ and $\hslice{\history}{j-1} = \hslice{\history}{\iini{1} \cap [1,j-1]}$, we have $t_{j-1} = s_{j-1}$ and $t_{j-1}' = s_{j-1}'$. Now since $\neg \theta_j(\tup_j)$ and $\neg \theta_j(\tup_{j}')$, $t_j = s_{j-1}$ and $t_j' = s_{j-1}'$. Furthermore, note that $\hslice{\history}{j+1,n} = \hslice{\history}{\iini{1} \cap [j+1,n]}$ and $\hslice{\ahmod}{j+1,n} = \hslice{\ahmod}{\iini{1} \cap [j+1,n]}$. Thus, it follows that $t = \tup_n = s_{n-1}$ (the sliced histories have one less update) and $\tup' = s_{n-1}'$ which contradicts $\tup \not\in \dslicei{1}$.

%%%%%%%%%%%%%%%%%%%%%%%%%%%%%%%%%%%%%%%%
\proofpar{$\tup \not\in \Delta \land \tup \in \dslicei{1}$:}
Since $\hslice{\history}{\iini{1}}$ (and $\hslice{\ahmod}{\iini{1}}$) are histories, from \Cref{theo:data-slicing} follows that if $s \in \dslicei{1}$ then $s_1 \models \theta_1 \vee \theta_1'$ ($s_i$, $t_i$, $s_i'$ and $t_i'$ are as defined above). Because we know that $\exclusion{\history}{\deltaHist}{u_j}$ we also know that $\neg \theta_j(s_{j-1})$ and $\neg \theta_j(s_{j-1}')$. Applying the same argument as for the opposite direction proven above, this implies that $s \in \Delta$ which contradicts $s \not \in \Delta$.

%%%%%%%%%%%%%%%%%%%%%%%%%%%%%%%%%%%%%%%%%%%%%%%%%%%%%%%%%%%%%%%%%%%%%%%%%%%%%%%%
\proofpar{Induction Step:}
Assume that $\iini{i}$ is a slice for $\hwhatif$, i.e., $\Delta = \dslicei{i}$. We have to show that $\iini{i+1}$ is also a slice, i.e., $\Delta = \dslicei{i+1}$. Note that $\dslicei{i+1}$ differs from $\dslicei{i}$ only in that it excludes an additional update at a position $j$, i.e., $\iini{i+1} = \iini{i} - \{ j \}$ and $\forall l \in \iexi{i}: j > l$. Let $k = \max(\iexi{i})$. The remainder of the argument proceeds analog to the base case. For sake of contradiction assume that there exists a tuple $\tup$ with $\tup \in \Delta \land \tup \not\in \dslicei{i+1}$. Let use again employ the notation $s_i$, $s_i'$, $t_i$ and $t_i'$ as above. Then by applying the argument from the base case iteratively, we can show that $t_{j-1} = s_{j-1}$ and $t_{j-1}'= s_{j-1}'$. Together with $\exclusion{\history}{\deltaHist}{u_j}$ this implies $t_j = s_{j-1}$ and $t_{j}' = s_{j-1}'$ (again using the same argument already applied in the base case) and in turn implies $\tup \in \dslicei{i+1}$. The proof for $\tup \not\in \Delta \land \tup \in \dslicei{i+1}$ is also analog to the base case and, thus, we omit it here.

% 


    % 
    % 
    
    $\,$\\[-3.5mm]
  \end{proof}
}

%%% Local Variables:
%%% mode: latex
%%% TeX-master: "techreport"
%%% End:
