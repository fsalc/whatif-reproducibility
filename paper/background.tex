
\section{Background and Notation}
\label{sec:background}


\begin{figure}[t]
\centering
\begin{minipage}{1.0\linewidth}
\begin{align*}
\exprbf                                & := \varbf | \cons | \exprbf \lbrace +,-,\times,\div\rbrace \exprbf | \sqlCase{\condbf}{\exprbf}{\exprbf} \\
\condbf                                & := \exprbf \lbrace =,\neq,<,\leq,>,\geq \rbrace \exprbf | \condbf \lbrace \wedge ,\vee \rbrace \condbf | \exprbf \isnull | \neg \condbf | \T | \F
\end{align*}
\end{minipage}\\[-5mm]
\caption{Syntax of expressions $\exprbf$ and conditions $\condbf$}
  \label{fig:expr-grammar}
\end{figure}
\iftechreport{

\begin{figure}[t]
\centering
\begin{align*}
\exprbf + \exprbf'                     & = \exprbf' + \exprbf & \exprbf \times \exprbf' & = \exprbf' \times \exprbf \tag{commutativity}
\end{align*}
\begin{align*}
\begin{split}
\exprbf + (\exprbf' + \exprbf'')       & = (\exprbf + \exprbf') + \exprbf''\\
\exprbf \mul (\exprbf' \mul \exprbf'') & = (\exprbf \mul \exprbf') \mul \exprbf''
  \end{split}  \tag{associtivity}
\end{align*}\\[-5mm]
  \caption{Equivalence rules for expressions \exprbf}
  \label{fig:equ-for-expr}
\end{figure}

}

Given a universal value domain $\dataDomain$, a relation $\rel$ (instance) of 
arity $n$ is a subset of $\dataDomain^{n}$. A database instance (or database for short) $\db$ 
is a set of relations $\rel_1$ to $\rel_n$. We use $\schema{\rel}$ to denote the schema of relation $\rel$.

We consider three type of update operations: updates, inserts, and deletes. In the following, we will use the term \textit{update statement}, or statement for short, as an umbrella term for updates, deletes, and inserts. We view statements as functions that take a relation $\rel$ (or database in the case of inserts with a query) as input and return an updated version of $\rel$. We use $\up$ to denote any such statement and use $\up(\rel)$ (and sometimes abusing notation also $\up(\db)$) to denote the result of applying statement $\up$ to relation $\rel$. An insert $\ainsert(\rel)$ inserts tuple $t$ with the same arity as $\rel$ into relation $\rel$. An insert $\aqinsert(\rel)$ inserts the result of the query $\query$ evaluated over database $\db$ into $\rel$. A delete $\adelete(\rel)$ removes all tuples from $\rel$ that do not fulfill condition $\cond$. Finally, an update $\aupdate(\rel)$ updates the values of each tuple $t$ that fulfills condition $\cond$ based on a list of expressions $\pset$ and returns all other input tuples unmodified. $\pset$ is a list of expressions $(\expr_1,...,\expr_n)$ with the same arity as $\rel$. Each such expression is over the schema of $\rel$. We will sometimes use $(\attr{i_1}\leftarrow\expr_1,...,\attr{i_m}\leftarrow\expr_m)$ as a notional shortcut assuming that the expression for each attribute that is not explicitly mentioned is the identity. For instance, $\pset = (B \leftarrow B + 3)$ over schema $(A,B,C)$ denotes $(A, B + 3, C)$. For an update or delete $\up$ we use $\condOf{\up}$ to denote the update's (delete's) condition. Similarly, $\psetOf{\up}$ for an update $\up$ denotes the update's list of $\pset$ expressions.





A condition $\cond$ (as used in updates and deletions) is a Boolean expression over 
comparisons between scalar expressions containing variables and constants. The grammar defining the syntax of $\pset$ and $\cond$ expressions is shown in \Cref{fig:expr-grammar}. For any expression $\expr$, $\expr'$, and $\expr''$ we use $\subst{\expr}{\expr'}{\expr''}$ to denote the result of substituting each occurrence of $\expr'$ in $\expr$ with $\expr''$. We write $\pset(t)$ to denote the tuple produced by evaluating the expressions from $\pset$ over input tuple $t$ (required to be of the same arty as $\pset$). For example, for a relation $R(A, B, C)$, tuple $t = (1, 1, 1)$, and  $\pset = (A, A + B, 20)$ we get $\pset(t) = (1, 2, 20)$. 
Sometimes, we will use $\up(t)$ to denote the tuple that is the result of applying a statement $\up$ to a single tuple $t$.
We formally define the semantics of evaluating statements over a database $\db$ below. Note that the update statements we define here correspond to SQL update and delete statements without nested subqueries and joins and to \lstinline!INSERT INTO ... VALUES ...! and \lstinline!INSERT INTO ... SELECT ...!.


\begin{align}
  \aupdate(R) &= \{ \pset(t) \mid t \in R \wedge \cond(t) \} \cup \{ t \mid t \in R \wedge \neg \cond(t) \} \label{eq:update-sem}\\
  \adelete(R) &= \{ t \mid t \in R \wedge \neg \cond(t) \} \label{eq:delete-sem}\\
  \ainsert(R) &= R \cup \{ t \} \label{eq:const-insert-sem}\\
  \aqinsert(R) &= R \cup \query(\db) \label{eq:query-insert-sem}
\end{align}




A \textbf{history} $\history = \up_1, \ldots, \up_n$ 
over a database $\db$ is a sequence of updates over $\db$. 


Given a history $\history = \up_1, \ldots,\up_n$, we use $\histslice{i}{j}$ for $i \leq j \in [1,n]$ to denote $u_i, u_{i+1}, \ldots, u_j$. Similarly, $\histpre{i}$, called a prefix of $\history$, denotes $\history_{1,i}$. Furthermore, for a set of indices $\idxs = \{ \idx_1, \ldots, \idx_m \}$ such that $\idx_j < \idx_k$ if $j < k$ and $\idx_j, \idx_k \in [1,n]$, we use $\hislice{\idxs}$ to denote $(\up_{\idx_1}, \ldots, \up_{\idx_m})$.

We use $\history(\db)$ to denote the result of evaluating the history $\history$ over a database instance $\db$ (recursively defined below using the fact that $\histpre{n} = \history$) and will use $\dbver{\idx}$ to denote $\histpre{\idx}(\db)$.

\begin{align*}
  \db_1 &= \up_1(\db) &\db_i &= \up_i(\db_{i-1}) \tag{for $1 < i \leq n$}
\end{align*}

Our program slicing technique relies on a property we call \emph{tuple independence}. Intuitively, statements that fulfill this property process each input tuple individually.


\begin{defi}[Tuple independence]\label{def:tuple-independence}
 A statement $\up$ is \emph{tuple independent} if for every database $\db$, we have $\up(\db) = \bigcup_{t \in \db} \up(\{t\})$
  
  
  
\end{defi}

\BG{We could allow one tuple per table instead and get tuple independence for inserts with conjunctive queries, or up to one tuple per union to get independence for union of conjunctive queries?}
In SQL, all updates and deletes without nested subqueries or joins and inserts without queries are tuple independent. Thus, all of our statements with the exception of $\aqinsert$ are tuple independent.


\begin{lem}[Tuple independent statements]\label{lem:tuple-independent-operator}
All updates $\aupdate$, deletes $\adelete$, and inserts $\ainsert$ are tuple independent.
\end{lem}

\ifnottechreport{
  \begin{proofsketch}
Proven by unfolding of definitions and using the fact that comprehension distributes over union if the conditions in the comprehension are only over the element that is returned. That is, for any set $S$ and condition $\psi$ that only depends on $e$, the following equivalence holds: $\{ e \mid e \in S \land \psi \} = \bigcup_{e \in S} \{ e \mid \psi \}$. For the full proof please see~\cite{techreport}
  \end{proofsketch}
}
\iftechreport{  \begin{proof}
    WLOG consider a database $\db$ containing tuples $\{s_1, \ldots, s_m\}$ and let $\{t_1, \ldots, t_n\}$ be the instance of the relation $R$ to which a statement is applied to. Note that for any set comprehension $\{ e \mid e \in S \land \psi \}$ where $S = \{e_1, \ldots, e_n\}$ is a set and $\psi$ is a condition over $e$, the following equivalence holds if $\psi$ does not reference $S$:

    %%%%%%%%%%%%%%%%%%%%%%%%%%%%%%%%%%%%%%%%
    \begin{align}
      \label{eq:factor-comprehensions}
      \{ e \mid e \in S \land \psi \} = \bigcup_{e \in S} \{ e \mid \psi \}
    \end{align}
    %%%%%%%%%%%%%%%%%%%%%%%%%%%%%%%%%%%%%%%%

    For deletes, updates, and insert of constant tuples ($\ainsert$), their result only depends on $\rel$ and no other relation in $\db$. Thus, they return $\emptyset$ for any single tuple instance $\{ s_i\}$ unless tuple $s_i$ belongs to $R$ and we trivially have for any statement $\up$ where $\up$ is either an update $\aupdate$, delete $\adelete$, or insert of a constant tuple $\ainsert$:

    %%%%%%%%%%%%%%%%%%%%%%%%%%%%%%%%%%%%%%%%
    \begin{align}
      \label{eq:rel-update-same-as-database-update}
      \bigcup_{t \in R}  \up(\{t\}) = \bigcup_{t \in \db} \up(\{t\})
    \end{align}
    %%%%%%%%%%%%%%%%%%%%%%%%%%%%%%%%%%%%%%%%

    %%%%%%%%%%%%%%%%%%%%%%%%%%%%%%%%%%%%%%%%%%%%%%%%%%%%%%%%%%%%
    \proofpar{Updates}:
    Consider an update $\aupdate$.
    %%%%%%%%%%%%%%%%%%%%%%%%%%%%%%%%%%%%%%%%
    \begin{align*}
      \aupdate(R) &= \{ \pset(t) \mid t \in R \wedge \cond(t) \} \cup \{ t \mid t \in R \wedge \neg \cond(t) \}\\
                  &= \bigcup_{t \in R} \{ \pset(t) \mid \cond(t) \} \cup
                    \bigcup_{t \in R} \{ t \mid \neg \cond(t) \} \Cref{eq:factor-comprehensions}\\
                  &= \bigcup_{t \in R} \{ \pset(t) \mid \cond(t) \} \cup \{ t \mid \neg \cond(t) \}\\
                  &= \bigcup_{t \in R}  \aupdate(\{t\}) \tag{\Cref{eq:update-sem}}\\
                  &= \bigcup_{t \in \db}  \aupdate(\{t\}) \tag{\Cref{eq:rel-update-same-as-database-update}}
    \end{align*}
    %%%%%%%%%%%%%%%%%%%%%%%%%%%%%%%%%%%%%%%%
    %%%%%%%%%%%%%%%%%%%%%%%%%%%%%%%%%%%%%%%%%%%%%%%%%%%%%%%%%%%%
    \proofpar{Deletes}
    Consider a delete $\adelete$.
    %%%%%%%%%%%%%%%%%%%%%%%%%%%%%%%%%%%%%%%%
    \begin{align*}
      \adelete(R) &= \{ t \mid t \in R \wedge \neg \cond(t) \} \\
                  &= \bigcup_{t \in R} \{ t \mid \cond(t) \} \tag{\Cref{eq:factor-comprehensions}}\\
                  &= \bigcup_{t \in R}  \adelete(\{t\}) \tag{\Cref{eq:delete-sem}}\\
                  &= \bigcup_{t \in \db}  \adelete(\{t\}) \tag{\Cref{eq:rel-update-same-as-database-update}}
    \end{align*}
    %%%%%%%%%%%%%%%%%%%%%%%%%%%%%%%%%%%%%%%%
    %%%%%%%%%%%%%%%%%%%%%%%%%%%%%%%%%%%%%%%%%%%%%%%%%%%%%%%%%%%%
    \proofpar{Inserts}
    Consider an insert $\ainsert(R)$.
    %%%%%%%%%%%%%%%%%%%%%%%%%%%%%%%%%%%%%%%%
    \begin{align*}
      \ainsert(R) &= \rel \union \{ t\}\\
                  &= \bigcup_{s \in R} \{ s \} \union \{ t\}\\
                  &= \bigcup_{s \in R} (\{ s \} \union \{ t\})\\
                  &= \bigcup_{s \in R} \ainsert(\{s\}) \tag{\Cref{eq:const-insert-sem}}\\
                  &= \bigcup_{s \in \db}  \ainsert(\{s\}) \tag{\Cref{eq:rel-update-same-as-database-update}}
    \end{align*}
    %%%%%%%%%%%%%%%%%%%%%%%%%%%%%%%%%%%%%%%%
    %%%%%%%%%%%%%%%%%%%%%%%%%%%%%%%%%%%%%%%%%%%%%%%%%%%%%%%%%%%%
    \proofpar{Inserts with queries}
    Inserts $\aqinsert(R)$ are not tuple independent. As a counterexample, consider $\up = \ins{\projection_{B,B}(R \join_{B=C} S)}(R)$ over $\rel(A,B)$ and $S(C)$ and database instance $R = \{(1,2)\}$ and $S = \{(2)\}$:
    \[
      \up(\db) = \{ (1,2), (2,2) \}
    \]
    while
    %%%%%%%%%%%%%%%%%%%%%%%%%%%%%%%%%%%%%%%%
    \begin{align*}
      \bigcup_{t \in \db} \up(\{t\}) &= \up(R=\{(1,2)\},S=\emptyset) \cup \up(R=\emptyset,S=\{(2)\}) = \{(1,2)\} \cup \emptyset\\
                                       &= \{(1,2)\}
    \end{align*}
    %%%%%%%%%%%%%%%%%%%%%%%%%%%%%%%%%%%%%%%%
    \end{proof}



%%% Local Variables:
%%% mode: latex
%%% TeX-master: "techreport"
%%% End:
}




\section{Historical What-if Queries}
\label{sec:whif-def}

We now formally define historical what-if queries. 

Let $\history$ be a history containing an update $\up$.
Historical what-if queries are based on \textbf{modifications} $\modi = \up \gets \up'$ that replace the statement $\up$ in $\history$ with another statement $\up'$, delete the statement $\up$ at position $i$ ($\modi = \mdel{i}$), or insert a new statement $\up$ at position $i$ ($\modi = \minsert{\up}{i}$).

We use $\deltaHist$ to denote a sequence of modifications and $\ahmod$ to denote the result of applying the modifications

$\deltaHist$ to the history $\history$. For example, for a history $\history = u_1, u_2, u_3$ and $\deltaHist = (u_1 \gets u_1', \mdel{3})$ we get $\ahmod = u_1', u_2$. 
Replacing a statement $\up$ with a statement $\up'$ of a different type, e.g., replacing an update with a delete, can be achieved by deleting $\up$ and then inserting $\up'$.


To answer a historical what-if query, we need to compute the difference between the current state of the database, i.e., $\history(\db)$ and the database produced by evaluating the modified history, i.e., $\ahmod(\db)$.
For that we introduce the notion of a database delta.
A \emph{database delta} $\iDiff{\db}{\db'}$ contains all tuples that only occur in $\db$ or in $\db'$. Tuples that exclusively are in $\db'$ are annotated with a $+$ and tuples that exclusively appear in $\db$ are annotated with $-$.  



  \begin{align*}
    \iDiff{\db}{\db'} &=
\{ +t \mid t \not\in \db \wedge t \in \db' \} \cup \{ -t \mid t \in \db \wedge t \not\in \db' \}



  \end{align*}


  We define a historical what-if query and an answer to such query based on the delta of $\history(\db)$ and $\ahmod(\db)$.


\begin{defi}[Historical What-If Queries]
A \textbf{historical what-if query} $\hwhatif$ is a tuple $(\history, \db, \deltaHist)$ where $\history$ is a  history executed over database instance $\db$, and $\deltaHist$ denotes a sequence of modifications to $\history$ as introduced above. 
The answer to $\hwhatif$ is defined as:
\begin{align*}
  \iDiff{\history(\db)}{\ahmod(\db)}

  \end{align*}
\end{defi}






















\begin{exam}
\label{ex:change-prob-example}
Let $\db$ and $\history$ be the database shown in \Cref{fig:running-example-instance} and history shown in \Cref{fig:updated-example-instance}, respectively. Consider the modification $\deltaHist_1 = (\up_1 \gets {\up_1}')$ where $\up_1$ and ${\up_1}'$ are the updates shown in \Cref{fig:Transitive-Transactions-Example}. $\deltaHist_1$ increases the minimum price for waving shipping fees. 
Bob's historical what-if query from this example can be written as $\hwhatif_{Bob} = (\history, \db, \deltaHist_1)$ in our framework.
Evaluating $\history[\deltaHist_1]$ results in the modified database instance shown in \Cref{fig:whatif-example-instance}. For convenience, we have highlighted modified tuple values. The answer of the \abbrHW $\hwhatif_{Bob}$ is
\[
  \iDiff{\history(\db)}{\history[\deltaHist_1](\db)} = \{ -o_6, +o_6'\}
\]


That is, the shipping fee for Alex's order is increased by \$5 because it is no longer eligible for free shipping under the new policy (${\up_1}'$).



\end{exam}



%%% Local Variables:
%%% mode: latex
%%% TeX-master: "historical_whatif"
%%% End:
