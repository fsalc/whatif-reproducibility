%%%%%%%%%%%%%%%%%%%%%%%%%%%%%%%%%%%%%%%%
\section{Data Slicing}
\label{sec:filter}

In this section, we present \textit{data slicing}, a technique which excludes data from reenactment for a \abbrHW $\hwhatif$ without affecting the result. Our technique is based on the observation that any difference between $\history(\db)$ and $\history[\deltaHist](\db)$ has to be caused  by a difference between $\history$ and $\history[\deltaHist]$. Thus, any tuple that is in the result of $\hwhatif$ has to be derived from a tuple that was affected (e.g., fulfills the condition of an update) by a statement affected by $\deltaHist$ in either the original history, the modified history, or both (but in different ways).

For example, in our running example from \Cref{fig:Transitive-Transactions-Example} the original update $\up_1$ and modified update ${\up_1}'$ only modify tuples for which either $Price \geq 50$ or $Price \geq 60$. For instance, the tuple with ID 11 does not fulfill any of these two conditions. Even though this tuple is modified by both histories, the same modifications are applied and, thus, the final result is the same (see \Cref{fig:updated-example-instance} and \Cref{fig:whatif-example-instance}): the shipping fee of this order was changed to \$8. Our data slicing technique determines selection conditions that filter out such tuples. For instance, for our running example we can apply the condition shown below (checking that either $\up_1$ or ${\up_1}'$ may modify the tuple):
%%%%%%%%%%%%%%%%%%%%%%%%%%%%%%%%%%%%%%%%
  \begin{align*}
    (Price \geq 50) \lor (Price \geq 60)
  \end{align*}
  %%%%%%%%%%%%%%%%%%%%%%%%%%%%%%%%%%%%%%%%
%
Initially, we will limit the discussion to data slicing for a single modification $\modi = \up \gets \up'$ where $\up$ and $\up'$ are of the same type (e.g., both are updates). We will show how to construct conditions $\condDSh{\modi}$ and $\condDSm{\modi}$ that we apply to filter irrelevant tuples from the inputs of $\ract{\history}$ and $\ract{\history[\deltaHist]}$. As explained above, for a single modification $\up \gets \up'$  we can assume WLOG that $\up$ is the first update in $\history$, because any update before $\up$ can be ignored for reenactment. Afterwards, we extend the technique for multiple modifications and modifications that insert or delete statements (which also covers modifications that replace a statement with a statement of a different type). In the following, we will use $\qDSh$ to denote $\selection_{\condDSh{\modi}}(\rel)$ and $\qDSm$ to denote $\selection_{\condDSm{\modi}}(\rel)$.


% for determining a condition that can be applied to the database that is input to the reenactment query for a historical what-if query to . For now consider a single modification $\deltaHist = \{m\}$ with $m = \up \gets \up'$ where both $\up$ and $\up'$ are of the same type (update, delete, or insert).

%
% Intuitively, any difference between $\history(\db)$ and $\history[\deltaHist](\db)$ has to be caused directly or indirectly by a difference between $\history$ and $\history[\deltaHist]$. % For simplicity of exposition consider a single modification
% For a tuple $t$ to be included in $\iDiff{\history(\db)}{\history[\deltaHist](\db)}$, it has to be affected by either $\up$ or $\up'$ or both (but in different ways). % There exist provenance models (e.g., the MV-semiring model~\cite{AG17}) that allow us to determine the relevant tuples, however, this would require reenacting $\history(\db)$ and $\ahmod(\db)$ with provenance ahead of time, defeating the purpose of our static optimizations.
% Instead

% We demonstrate how to construct a condition that can be applied to the input database such that any tuple filtered out by this condition is guaranteed to not contribute to $\iDiff{\history(\db)}{\history[\deltaHist](\db)}$.
% First, consider a modification $\modi = \up \gets \up'$. We will show how to construct conditions $\condDSh{\modi}$ and $\condDSm{\modi}$ that we apply to filter the inputs of $\ract{\history}$ and $\ract{\history[\deltaHist]}$ to filter out tuples that cannot contribute to the result of the historical what-if query. In the following, we will use $\qDSh$ to denote $\selection_{\condDSh{\modi}}(\rel)$ and $\qDSm$ to denote $\selection_{\condDSm{\modi}}(\rel)$.

%%%%%%%%%%%%%%%%%%%%%%%%%%%%%%%%%%%%%%%%%%%%%%%%%%%%%%%%%%%%%%%%%%%%%%%%%%%%%%%%
\partitle{Updates}
First, consider a modification $\modi = \up \gets \up'$ where both $\up$ and $\up'$ are updates.
Since only tuples that match the condition of an update operation (the operation's \lstinline!WHERE! clause) can be affected by the operation, a conservative overestimation of $\iDiff{\history(\db)}{\history[\deltaHist](\db)}$ is the set of tuples that are derived from tuples affected by $\up$ in the original history or $\up'$ in the modified history.
Thus, the tuples in $\db$ from which such a tuple is derived have to either match the condition of $u$ ($\condOf{u}$) or the condition of $u'$ ($\condOf{u'}$). This means we can filter the input to the reenactment queries using:
%
\begin{align}
\condDSh{\modi} = \condDSm{\modi} = \condOf{u} \vee \condOf{u'} \label{eq:update-ds-cond}
\end{align}
%
% \BG{Applying a selection to a database is abusing notation, we need to at least explain what we mean by that or alternatively go into more depth of how we filter (we apply the selection condition to the table affected by the updates (assuming that they affect the same table) and then  . Also probably better to using $\db'$ or similar since the result is again a database.}
% We can filter input data for generated reenactment queries by considering the input data for the original updates and modified updates. As $\qResultDiff{\query}{\history(\db)}{\history[\deltaHist](\db)}$ includes tuples that are modified by the original updates or modified updates, we can limit the input data to reenactment queries to only these tuples.

%%%%%%%%%%%%%%%%%%%%%%%%%%%%%%%%%%%%%%%%%%%%%%%%%%%%%%%%%%%%%%%%%%%%%%%%%%%%%%%%
\partitle{Deletes}
Let us now consider a single modification $\up \gets \up'$ which replaces a delete $\up = \delete{\cond}$ with a delete $\up' = \delete{\cond'}$. For a tuple $\tup \in \rel$ to contribute to $\iDiff{\ract{\history}(\rel)}{\ract{\history[\deltaHist]}(\rel)}$, it has to be deleted by either $\up$ or $\up'$, but not by both (such tuples do not contribute to any result of $\ract{\history}(\rel)$ or $\ract{\history[\deltaHist]}(\rel)$). Thus, we can filter from $\rel$ all tuples that do not fulfill the condition
%
\begin{align}
\condDSh{\modi} = \condDSm{\modi} = (\cond \land \neg\,\cond') \vee (\neg\,\cond \land \cond')\label{eq:ds-del-cond}
\end{align}
%
\iftechreport{Note that for any tuple $\tup_{out}$ to be in the result of $\ract{\history}(\rel)$ ($\ract{\history[\deltaHist]}(\rel)$), it has to be the case that the input tuple $\tup$ in $\rel$ it is derived from has to not fulfill the condition of $\up$ ($\up'$), otherwise $\tup$ would have been deleted. That is, for $\history$, any tuple fulfilling $(\cond \land \neg\,\cond')$ will be filtered out by the delete. Similarly, for $\ahmod$, any tuple fulfilling $(\neg\,\cond \land \cond')$ will be deleted. Thus, we can simplify the data slicing conditions from \Cref{eq:ds-del-cond} by removing this redundant test and get:
%%%%%%%%%%%%%%%%%%%%%%%%%%%%%%%%%%%%%%%%
  \begin{align*}
    \condDSh{\modi} &= \neg\,\condOf{\up} \land \condOf{\up'}\\
    \condDSm{\modi} &= \condOf{\up} \land \neg\,\condOf{\up'}
  \end{align*}
  %%%%%%%%%%%%%%%%%%%%%%%%%%%%%%%%%%%%%%%%
%
  Furthermore, for any tuple $t$ ``surviving'' the delete of $\history$ ($\ahmod$) we have that $t$ fulfills the condition $\neg\,\condOf{\up}$ ($\neg\, \condOf{\up'}$). This means the conditions can be further simplified:
%
%%%%%%%%%%%%%%%%%%%%%%%%%%%%%%%%%%%%%%%%
  \begin{align*}
    \condDSh{\modi} &= \condOf{\up'}\\
    \condDSm{\modi} &= \condOf{\up}
  \end{align*}
  %%%%%%%%%%%%%%%%%%%%%%%%%%%%%%%%%%%%%%%%
  % Thus, we can simplify \Cref{eq:ds-del-cond} to $\cond'$ for $\ract{\history}$ and to $\cond$ for $\ract{\history[\deltaHist]}$.
}

%%%%%%%%%%%%%%%%%%%%%%%%%%%%%%%%%%%%%%%%%%%%%%%%%%%%%%%%%%%%
\partitle{Inserts with Queries}
Recall that an insert $\aqinsert$ is reenacted using the query $\rel \union \query$. Only tuples that are returned by the query $\query$ need to be considered. Thus, if $\aqinsert$ is the only statement that is modified, then it is sufficient to replace $R \union \query$ in the reenactment query with $\query$. However, for multiple modifications, tuples from the LHS of the union of the reenactment query for a statement $\aqinsert$ may be affected by downstream updates modified by $\deltaHist$. Thus, we cannot simply replace $R \union \query$ with $\query$ if $\aqinsert$ is not the first and only statement in the history that got modified by $\deltaHist$. To deal with this case, we need a condition that selects tuples which may contribute to the result of $\query$. We can achieve this by pushing the selection conditions of $\query$ down to the relations accessed by $\query$. For that we apply standard selection move-around techniques from query optimization. The final result is a selection condition for each input relation of the query. For instance, for $\ins{\selection_{A=5}(R \join_{A=C} S)}(R)$ over relations $R(A,B)$ and $S(C,D)$, the selection can be pushed to both inputs of the join resulting in a condition $A=5$ for $R$ and $C=5$ for $S$.
We formally define the rules for pushing conditions through queries in \cite{techreport}.
% Data slicing for delete statements are different from update statements as they remove some tuples from the database and we do not need to consider tuples that are deleted by both the original and modified statements. If a tuple is removed by both original and modified statements, it would not be in the result of $\ract{\history}$ nor $\ract{\history[\deltaHist]}(\db)$ reenactment queries, so we can filter them in computing the delta without any changes to the result of historical what-if query. To compute the difference between the result of $\ract{\history}(\db)$ and $\ract{\history[\deltaHist]}(\db)$, we need tuples that are deleted by the original delete statement but not by modified one or vice versa.

%%%%%%%%%%%%%%%%%%%%%%%%%%%%%%%%%%%%%%%%%%%%%%%%%%%%%%%%%%%%%%%%%%%%%%%%%%%%%%%%
\partitle{Multiple modifications}
Data slicing can also be applied to \abbrHWs with more than one modification. For a tuple to be in the result of the what-if query, it has to be affected by at least one statement $\up$ such that there exists one modification $\modi \in \deltaHist$ with either $\modi = \up \gets \up'$ or $\modi = \up' \gets \up$ for some statement $\up'$. However, we cannot simply use the disjunction of the data slicing conditions $\condDSh{\modi}$ and $\condDSm{\modi}$ we have developed for single modifications to filter the input. To see why this is the case, consider a modification $\modi = \up \gets \up'$ where $\up$ is the $i^{th}$ update in $\history$. The
input of $\up$ ($\up')$ over which the condition of the update is evaluated is the result of $\hislice{i-1}$ (or $\hisliceOf{\history[\deltaHist]}{i-1}$). To be able to derive a selection condition that can be applied to $\rel$, we have to ``push'' the condition for $\up$ down to determine a condition that returns the set of tuples from $\rel$ that contribute to tuples in $\hislice{i-1}$ fulfilling condition $\condDSh{\modi}$ (or $\condDSm{\modi}$).
For that, we iteratively substitute references to attributes in $\condDSh{\modi}$ (or $\condDSm{\modi}$) with the expressions from the previous statement in $\history$ that defines them. For instance, consider a history $\history = (\up_1 = \update{A \gets 3}{C = 5}, \up_2 = \update{B \gets B + 1}{A < 4})$ and modification $\modi = \up_2 \gets \up_2'$ with $\up_2' = \update{B \gets B + 1}{A < 5}$. To push the condition $A < 4$ of $\up_2$, we substitute $A$ with $\sqlCase{C=5}{3}{A}$ and get $(\sqlCase{C=5}{3}{A}) < 4$.


More formally, consider a modification $\modi = \up_i \gets {\up_i}'$ for a history $\history = (\up_1, \ldots, \up_n)$. Let us first consider how to push $\condDSh{\modi}$ (the case for $\condDSm{\modi}$ is symmetric).
We construct $\pushCond{\condDSh{\modi}}{j}$, the version of $\condDSh{\modi}$ pushed down through $j < i$ updates as shown below. We use $\pushCond{\condDSh{\modi}}{\ast}$ to denote $\pushCond{\condDSh{\modi}}{i-1}$, i.e., pushing the condition through all updates of the history before $\up$. Furthermore, we use an operator $\qpushCond{\cond}{\query}$  to push a condition $\cond$ through a query $\query$. See \cite{techreport} for the formal definition of this operator.
%
\begin{align*}
  \pushCond{\condDSh{\modi}}{0}                                                 & = \condDSh{\modi}                            \\
  \pushCond{\condDSh{\modi}}{j+1}                                               & =
                                    \begin{cases}
                                      \subst{\pushCond{\condDSh{\modi}}{j}}{\vec{A}}{\vec{e}} & \mathbf{if}\, \up_{i-j} = \update{\pset}{\cond} \\
                                      \pushCond{\condDSh{\modi}}{j} \lor \qpushCond{\pushCond{\condDSh{\modi}}{j}}{\query} &\mathbf{if}\, \up_{i-j} = \aqinsert\\
                                      \pushCond{\condDSh{\modi}}{j}             & \mathbf{otherwise}                           \\
                                    \end{cases}
\end{align*}
In the above equation, $\vec{A}$ denotes $(A_1, \ldots, A_n)$ and $\vec{e}$ denotes
%
\[
  (\sqlCase{\cond}{\pset(A_1)}{A_1},\ldots, \sqlCase{\cond}{\pset(A_n)}{A_n})
\]
Furthermore, $e[\vec{A} \gets \vec{e}]$ denotes the result of substituting each reference to $A_i$ in $e$ with $e_i$ (for all $i \in [1,n]$).

%%%%%%%%%%%%%%%%%%%%%%%%%%%%%%%%%%%%%%%%%%%%%%%%%%%%%%%%%%%%%%%%%%%%%%%%%%%%%%%%
\iftechreport{
The operator $\qpushCond{\cond}{\query}$ mentioned above pushes a selection condition through a query $\query$. So far we have assumed for easy of presentation that a history accesses a single relation $R$. We will stick to this restriction for now and define $\qpushCond{\cond}{\query}$ under this assumption. Afterwards, we will discuss how to generalize data slicing to histories that access multiple relations which is often the case for inserts that use queries.

  %%%%%%%%%%%%%%%%%%%%%%%%%%%%%%%%%%%%%%%%
  \begin{align*}
    \qpushCond{\cond}{R} &= \cond\\
    \qpushCond{\cond}{\selection_{\cond'}(\query)} &= \qpushCond{\cond \land \cond'}{\query}\\
    \qpushCond{\cond}{\projection_{\vec{e}}(\query)} &= \qpushCond{\cond[\vec{A} \gets \vec{e}]}{\query}\\
    % \qpushCond{\cond}{\aggregation_{f(A),G}(\query)} &= \qpushCond{\cond[\vec{A} \gets \vec{e}]}{\query}\\
    \qpushCond{\cond}{\query_1 \union \query_2} &= \qpushCond{\cond}{\query_1} \lor \qpushCond{\cond[\schema{\query_1} \gets \schema{\query_2}]}{\query_2}\\
    % \qpushCond{\cond}{\query_1 \join_{\cond'} \query_2} &= \qpushCond{\cond}{\query_1} \lor \qpushCond{\cond}{\query_2}\\
  \end{align*}
  %%%%%%%%%%%%%%%%%%%%%%%%%%%%%%%%%%%%%%%%

%%%%%%%%%%%%%%%%%%%%%%%%%%%%%%%%%%%%%%%%%%%%%%%%%%%%%%%%%%%%%%%%%%%%%%%%%%%%%%%%
\partitle{Data slicing for histories accessing multiple relations}
To generalize data slicing to histories that access multiple relations, we have to generate a separate slicing condition for every relation accessed by the history. For that we extend our push-down rules for conditions. Note that similar to how we deal with inserting and deleting statements from a history and replacing a statement $\up$ with a statement $\up'$ of a different type, modifications that change what relation is modified by a statement can be rewritten into a deletion of the original statement followed by a insertion of the modified statement. In turn these modifications can be rewritten into modifications that replace a statement with a statement of the same type that modifies the same relation using no-op statements. Thus, from now on we only need to consider modifications that replace a statement $\up$ with a statement $\up'$ where both $\up$ and $\up'$ modify the same relation. We use $\mpushCond{\cond}{\history}{\rel}$ to denote the condition generated for relation $\rel$ by pushing condition $\cond$ through history $\history$. Intuitively, statements that modify a relation $S$ can be ignored when computing the condition for a relation $\rel$ if $\rel \neq S$. For inserts with query, we use $\mqpushCond{\cond}{\query}{\rel}$, explained below, to push $\cond$ through the query $\query$ for relation $\rel$. The relation-specific data slicing conditions for updates, deletes, and inserts are shown below. As before we assume a modification $\up \gets \up'$ and use $\condOf{\up}$ to denote the condition of statement  $\up$ if $\up$ is an update or delete.

\begin{itemize}
\item Update $\aupdate(\rel)$:
  %%%%%%%%%%%%%%%%%%%%%%%%%%%%%%%%%%%%%%%%
  \begin{align*}
    \mcondDSh{\modi}{S} = \condDSm{\modi}{S} =
    \begin{cases}
      \condOf{u} \vee \condOf{u'} &\mathbf{if}\, \rel = S\\
      \T & \mathbf{otherwise}\\
    \end{cases}
  \end{align*}
  %%%%%%%%%%%%%%%%%%%%%%%%%%%%%%%%%%%%%%%%
\item Delete $\adelete(\rel)$:
  %%%%%%%%%%%%%%%%%%%%%%%%%%%%%%%%%%%%%%%%
  \begin{align*}
    \condDSh{\modi} &=
                      \begin{cases}
                        \condOf{u'} &\mathbf{if}\, \rel = S\\
                        \T & \mathbf{otherwise}\\
                      \end{cases}\\
    \condDSm{\modi} &=
                      \begin{cases}
                        \condOf{u} &\mathbf{if}\, \rel = S\\
                        \T & \mathbf{otherwise}\\
                      \end{cases}
  \end{align*}
  %%%%%%%%%%%%%%%%%%%%%%%%%%%%%%%%%%%%%%%%
% \item Insert $\ainsert(\rel)$:
\end{itemize}

Based on these extended definitions, we then define pushing relation-specific conditions through histories as shown in \Cref{fig:pushing-relation-specific}.

%%%%%%%%%%%%%%%%%%%%%%%%%%%%%%%%%%%%%%%%
\begin{figure*}[t]
  \centering
%%%%%%%%%%%%%%%%%%%%%%%%%%%%%%%%%%%%%%%%
\begin{align}
  \label{eq:pushing-queries-multi-rel}
  \mpushCond{\condDSh{\modi}}{0}{\rel}                                                 & = \mcondDSh{\modi}{\rel}                            \\
  \mpushCond{\condDSh{\modi}}{j+1}{rel}                                               & =
                                                                                        \begin{cases}
                                                                                          \subst{\mpushCond{\condDSh{\modi}}{j}{\rel}}{\vec{A}}{\vec{e}} & \mathbf{if}\, \up_{i-j} = \update{\pset}{\cond}(\rel) \\
                                                                                          \revm{\mpushCond{\condDSh{\modi}}{j}{\rel} \lor \mqpushCond{\mpushCond{\condDSh{\modi}}{j}{\rel}}{\query}{S}} &\revm{\mathbf{if}\, \up_{i-j} = \aqinsert}\\
                                                                                          \mpushCond{\condDSh{\modi}}{j}{\rel}             & \mathbf{otherwise}                           \\
                                                                                        \end{cases}
\end{align}
%%%%%%%%%%%%%%%%%%%%%%%%%%%%%%%%%%%%%%%%
  \caption{Pushing relation-specific data slicing conditions}\label{fig:pushing-relation-specific}
\end{figure*}
%%%%%%%%%%%%%%%%%%%%%%%%%%%%%%%%%%%%%%%%

Note that in the definition of $\mpushCond{\condDSh{\modi}}{j}{\rel}$ we make use of $\mqpushCond{\mpushCond{\condDSh{\modi}}{j}{\rel}}{\query}{S}$ which we define below. % Here, let $\relsOf{\query}$ denote the relations accessed by query $\query$.

  %%%%%%%%%%%%%%%%%%%%%%%%%%%%%%%%%%%%%%%%
  \begin{align*}
    \mqpushCond{\cond}{R}{S} &=
                               \begin{cases}
                                 \cond &\mathbf{if}\, R = S\\
                                 \T &\mathbf{otherwise}
                               \end{cases}\\
    \mqpushCond{\cond}{\selection_{\cond'}(\query)}{\rel} &= \mqpushCond{\cond \land \cond'}{\query}{\rel}\\
    \mqpushCond{\cond}{\projection_{\vec{e}}(\query)}{\rel} &= \mqpushCond{\cond[\vec{A} \gets \vec{e}]}{\query}{\rel}\\
    \mqpushCond{\cond}{\query_1 \union \query_2}{\rel} &= \mqpushCond{\cond}{\query_1}{\rel} \lor \mqpushCond{\cond[\schema{\query_1} \gets \schema{\query_2}]}{\query_2}{\rel}\\
  \end{align*}
  %%%%%%%%%%%%%%%%%%%%%%%%%%%%%%%%%%%%%%%%



%%% Local Variables:
%%% mode: latex
%%% TeX-master: "techreport"
%%% End:

}
%%%%%%%%%%%%%%%%%%%%%%%%%%%%%%%%%%%%%%%%%%%%%%%%%%%%%%%%%%%%%%%%%%%%%%%%%%%%%%%%

%%%%%%%%%%%%%%%%%%%%%%%%%%%%%%%%%%%%%%%%
\begin{exam}\label{ex:ds-example}
Consider our running example history and a modification that replaces $u_3$ (reducing shipping fee by \$2 if the shipping fee is at least \$10 and the order price is at most \$30) with ${\up_3}'$ which applies to orders of $\leq \$40$: ${\up_3}' = \update{F \gets F-2}{P \leq 40 \land F \geq 10}$.
%%%%%%%%%%%%%%%%%%%%%%%%%%%%%%%%%%%%%%%%
% \begin{align*}
% {\up_2}' = \update{F \gets F-2}{P \leq 40 \land F \geq 10}
% \end{align*}
%%%%%%%%%%%%%%%%%%%%%%%%%%%%%%%%%%%%%%%%
The data slicing condition for $u_3$ and ${\up_3}'$ is $(P \leq 30 \land F \geq 10) \vee (P \leq 40 \land F \geq 10)$ which can be simplified to $(P \leq 40 \land F \geq 10)$. To push this condition through $\up_2$, we have to substitute $F$ (the shipping fee) with the conditional update of the shipping fee corresponding to $\up_2$ and get
$(P \leq 40 \land F'' \geq 10)$ for $F'' = \sqlCase{C=UK \land P \leq 100}{F+5}{F}$. We then have to push this condition through $\up_1$. For that we substitute $F$ again, this time with $F' = \sqlCase{P \geq 50}{0}{F}$. The final data slicing condition for both $\history$ and $\ahmod$ and our modification $\modi = \up_3 \gets {\up_3}'$ is:
%
%%%%%%%%%%%%%%%%%%%%%%%%%%%%%%%%%%%%%%%%
\begin{align*}
\pushCond{\condDSh{\modi}}{\ast} &= \pushCond{\condDSm{\modi}}{\ast} = (P \leq 40 \land F'' \geq 10)\\[1mm]
F'' &= \sqlCase{C=UK \land P \leq 100}{F'+5}{F'}\\
  F' &= \sqlCase{P \geq 50}{0}{F}
\end{align*}
%%%%%%%%%%%%%%%%%%%%%%%%%%%%%%%%%%%%%%%%
Evaluating this condition over the database from \Cref{fig:running-example-instance}, only the tuple with ID 11 has a sufficiently low price $P \leq 40$ and fulfills the condition $F'' \geq 10$ ($F = F' = 5$ and $F'' = F'+5 = 10$). Thus, using this slicing condition we can exclude tuples 12, 13, and 14 from reenactment.
% To compute $\iDiff{\ract{\history}(\db)}{\ract{\history[\deltaHist]}(\db)}$ for the \Cref{ex:reenact-example}, we first construct reenactment queries for updates $u_1, u_2, u_3$ and ${u_1}',u_2,u_3$. Based on the conditions of $u_1$ and ${u_1}'$, the data slicing condition we construct is $p \geq 40 \vee p \geq 50$.
% % we add a condition \lstinline!Price>=40 OR Price>=50! to the inputs of these queries (adding a \lstinline!WHERE! clause).
%   In our example, this conditions excludes tuples $o_1$ and $o_4$ from reenactment since any tuple in $\ract{\history}(\db)$ and $\ract{\history[\deltaHist]}(\db)$ derived from these tuples will occur in the result of both histories and, thus, cannot not be part of the what-if query's result. % So, we need to compute $\qResultDiff{\query}{\subract{O}{\history}(\rel)}{\subract{O}{\history[\deltaHist]}(\rel)}$ where $\rel\leftarrow \selection_{p>=40 \vee p>=50}(\db)$. for answering the historical what-if query.
% % \begin{lstlisting}
% % SELECT * FROM  $\ract{Employee[\modi]}$ M
% % WHERE Calls<30 OR Calls<50;
% % SELECT * FROM  $\ract{Employee}$ E
% % WHERE Calls<30 OR Calls<50;
% % \end{lstlisting}
% $\,$\\[-4mm]
\end{exam}
%%%%%%%%%%%%%%%%%%%%%%%%%%%%%%%%%%%%%%%%



%%%%%%%%%%%%%%%%%%%%%%%%%%%%%%%%%%%%%%%%%%%%%%%%%%%%%%%%%%%%%%%%%%%%%%%%%%%%%%%%
\partitle{Modifications that insert or delete statements}
Recall that we also allow modifications that insert a new statement at position $i$ ($\minsert{\up}{i}$) or delete the statement at position $i$ ($\mdel{i})$. Note that it is possible to insert new statements into a history without changing its semantics as long as these statements do not modify any data, e.g., a delete $\delete{\F}$ that does not delete any tuples. We refer to such operations as \emph{no-ops}. Using no-ops, we can pad the original history at position $i$ for every insert $\minsert{\up}{i}$. We then can rewrite  $\minsert{\up}{i}$ in $\deltaHist$ into a modification $\up_i \gets \up$ where $\up_i$ is a no-op. A deletion $\mdel{i}$ is rewritten into a modification $\up_i \gets {\up_i}'$ where ${\up_i}'$ is a no-op. Thus, the data slicing method explained above is already sufficient for dealing with inserts $\minsert{\up}{i}$ and deletes $\mdel{i}$.

%%%%%%%%%%%%%%%%%%%%%%%%%%%%%%%%%%%%%%%%%%%%%%%%%%%%%%%%%%%%%%%%%%%%%%%%%%%%%%%%
\begin{theo}[Data Slicing]\label{theo:data-slicing}
  Consider a history $\history$, a relation $R$, and a sequence of modifications $\deltaHist = (\modi_1, \ldots, \modi_n)$. Let $\qDSh = \selection_{\bigvee_{i=1}^{n} \pushCond{\condDSh{\modi_i}}{\ast}}(\rel)$
and $\qDSm = \selection_{\bigvee_{i=1}^{n} \pushCond{\condDSm{\modi_i}}{\ast}}(\rel)$.
  Then,
\[
  \iDiff{\ract{\history}(\rel)}{\ract{\history[\deltaHist]}(\rel)}
  =
\iDiff{\ract{\history}(\qDSh(\rel))}{\ract{\history[\deltaHist]}(\qDSm(\rel))}
\]
\end{theo}
%%%%%%%%%%%%%%%%%%%%%%%%%%%%%%%%%%%%%%%%%%%%%%%%%%%%%%%%%%%%%%%%%%%%%%%%%%%%%%%%
\ifnottechreport{
  \begin{proofsketch}
    We prove this theorem by induction over the number of modifications in $\deltaHist$. For the base case we prove the claim by case analysis (update and deletes). We show that any tuple not fulfilling $\cond$ and $\cond'$ does not contribute to the delta and, therefore can be excluded.  For the inductive step,  we prove by induction over the number of steps a conditions has to be "pushed-down", that the pushed-down condition ($\pushCond{\condDSh{\modi_i}}{\ast}$ or $\pushCond{\condDSm{\modi_i}}{\ast}$) excludes only irrelevant tuples. For the full proof see \cite{techreport}.
  \end{proofsketch}
}
\iftechreport{%%%%%%%%%%%%%%%%%%%%%%%%%%%%%%%%%%%%%%%%
% 
% 
% 
% 
% 
% 
% 
% 
% 
% 
% 
% 
% 
% 
% 
% 
% 
%%%%%%%%%%%%%%%%%%%%%%%%%%%%%%%%%%%%%%%%

%%%%%%%%%%%%%%%%%%%%%%%%%%%%%%%%%%%%%%%%
\newcommand{\tupin}{\tup_{in}}
\newcommand{\tupup}{\tup_{up}}
\newcommand{\dfull}{\Delta}
\newcommand{\dslice}{\Delta_{DS}}
%%%%%%%%%%%%%%%%%%%%%%%%%%%%%%%%%%%%%%%%

\begin{proof}
  We proof the theorem by induction over the number of modifications ($\card{\deltaHist}$).

%%%%%%%%%%%%%%%%%%%%%%%%%%%%%%%%%%%%%%%%%%%%%%%%%%%%%%%%%%%%%%%%%%%%%%%%%%%%%%%%
  \proofpar{\bf Base Case:}
  We consider a history $\history$ with a single modification $\up \gets \up'$ that replaces the first update of $\history$.
  In the following, we use $\dfull$ to denote
  \[
    \iDiff{\ract{\history}(\rel)}{\ract{\history[\deltaHist]}(\rel)}
  \]
  and $\dslice$ to denote
  \[
    \iDiff{\ract{\history}(\qDSh(\rel))}{\ract{\history[\deltaHist]}(\qDSm(\rel))}
    \]
    Note that histories $\history = (\up, \up_2, \ldots, \up_n)$ and $\history[\deltaHist]$ only differ in their first operation ($\up$ or $\up'$). We proof the claim first for updates ($\up = \aupdate$) and then for deletes ($\up = \adelete$).

%%%%%%%%%%%%%%%%%%%%%%%%%%%%%%%%%%%%%%%%%%%%%%%%%%%%%%%%%%%%%%%%%%%%%%%%%%%%%%%%
\proofpar{$\up = \aupdate$:}
%
For any tuple $\tup$ to be in $\dfull$, there has to exist a tuple $\tupup$ for which either (a)  $\histslice{2}{n}(\{\tupup\}) = \{\tup\}$ or (b) $\hsliceOf{\ahmod}{2}{n}(\{\tupup\}) = \{\tup\}$, i.e., $\tup$ is the result of applying one the histories excluding the first statement ($\up$ or $\up'$), and for which also either (i) $\tupup \in \up(\rel) \land \tupup \not\in \up'(\rel)$ or (ii) $\tupup \not\in \up(\rel) \land \tupup \in \up'(\rel)$ holds. To see why this has to be true, consider the only two remaining cases: for all tuples fulfilling (a) or (b) either (iii) $\tupup \in \up(\rel) \land \tupup \in \up'(\rel)$ or (iv) $\tupup \not\in \up(\rel) \land \tupup \not\in \up'(\rel)$ holds. In case (iii), the same suffix history $(\up_2, \ldots, \up_n)$ is applied to $\tupup$ in both $\history$ and $\history[\deltaHist]$ which means that $\tup \in \history(\rel)$ and $\tup \in \history[\deltaHist](\rel)$ which contradicts $\tup \in \dfull$. Case (iv) also contradicts the assumption that  $\tup \in \dfull$.
  Let us now only consider case (i) since (ii) is symmetric. Consider any tuple $\tupin \in \rel$ such that $\up(\tupin) = \tupup$ (recall that we use $\up(t) = t'$ as a notational shortcut for $\up(\{t\}) = \{t'\}$). We know that $\up'(\tupin) \neq \tupup$, because $\tupup \not\in \up'(\rel)$. This can only be the case if $\tupin$ fulfills the condition of update $\up$ and/or $\up'$, because if $\tupin$ does not fulfill the condition of either update, then both updates return $\tupin$ unmodified and we get $\up(\tupin) = \up(\tupin) = \tupin$ contradicting $\tupup \not\in \up'(\rel)$.

 Recall that $\condOf{\up}$ and $\condOf{\up'}$  are the conditions of $\up$ and $\up'$, respectively.  So far we have established that for any tuple $\tup$ in the result of either $\history$ or $\history[\deltaHist]$ and the tuple $\tupin \in \rel$ it is derived from by either $\history$ or $\history[\deltaHist]$, we have

 \begin{align}
\tup \in \dfull \Rightarrow \tupin \models \condOf{\up} \lor \condOf{\up'}
 \end{align}

Using the equivalence $\psi_1 \Rightarrow \psi_2 \Leftrightarrow \neg \psi_2 \Rightarrow \neg \psi_1$ we get:

%%%%%%%%%%%%%%%%%%%%%%%%%%%%%%%%%%%%%%%%
 \begin{align}
   \tupin \not\models \condOf{\up} \lor \condOf{\up'} \Rightarrow \tup \not\in \dfull
    \label{eq:update-slicing-cond}
 \end{align}
 %%%%%%%%%%%%%%%%%%%%%%%%%%%%%%%%%%%%%%%%

Since $\query_{DS}$ only filters out tuples $\tupin$ for which $\tupin \not \models \condOf{\up} \lor \condOf{\up'}$,  \Cref{eq:update-slicing-cond} implies that all tuples filtered by $\query_{DS}$ do not contribute to any tuple in $\dfull$. Thus, we get $\dfull = \dslice$ which concludes the proof for this case.


%
%
%
%
%
%
%

%

%
%
%
%
%

%%%%%%%%%%%%%%%%%%%%%%%%%%%%%%%%%%%%%%%%
\proofpar{$\up = \adelete$:}
%
%
Now consider the case where $\up$ is a delete statement.


%%%%%%%%%%%%%%%%%%%%%%%%%%%%%%%%%%%%%%%%
\proofpar{$\dfull \subseteq \dslice$:}
%
  We prove this direction by contradiction. We have two histories $\history$ and $\history[\deltaHist]$ that only differ in the first statement: $\up = \delete{\cond}$ in $\history$ and $\up' = \delete{\cond'}$ in $\history[\deltaHist]$.
%
  Consider  a tuple $\tup \in   \iDiff{\ract{\history}(\rel)}{\ract{\history[\deltaHist]}(\rel)}$ and WLOG assume the $\tup \in \ract{\history}(\rel)$ (the other case is symmetric). Then there has to exist $\tupin \in \rel$ such that $\ract{\history}(\{\tupin\}) = \{ \tup \}$, i.e., $\tupin$ is in the provenance of $\tup$. For this to be the case $\tupin \models \neg\,\cond$, i.e.,  $\tupin$ does not fulfill the condition $\cond$ of the delete $\up$ (otherwise it would have been deleted), and $\tupin \models \cond'$ (otherwise $\tup$ would not be in $\iDiff{\ract{\history}(\rel)}{\ract{\history[\deltaHist]}(\rel)}$).
  For sake of contradiction assume that $\tup \not \in \iDiff{\ract{\history}(\qDSh(\rel))}{\ract{\history[\deltaHist]}(\qDSm(\rel))}$. Since the two histories only differ in the first statement, this means that $\tupin$ does not fulfill the selection condition of $\qDSh$. Recall that this selection condition is $\cond'$. Thus, we have $\tupin \models \neg \cond'$ which contradicts $\tupin \models \cond'$.

%%%%%%%%%%%%%%%%%%%%%%%%%%%%%%%%%%%%%%%%
\proofpar{$\dfull \supseteq \dslice$:}
%
Consider a tuple $\tup \in \dslice$ and as above let $\tupin \in \rel$ denote the tuple it is derived from. We have to show that $\tup \in \dfull$. Since $\tup \in \dslice$ either $\tup \models \cond \land \tup \not\models \cond'$ or $\tup \not\models \cond \land \tup \models \cond$. Since these two cases are symmetric, WLOG assume that $\tup \models \cond \land \tup \not\models \cond$. Note that the only difference between  $\ract{\history}(\rel)$ and $\ract{\history}(\qDSh(\rel))$ is the selection applied by $\qDSh$. Thus, $\tup \in \ract{\history}(\qDSh(\rel))$. Also $\tup \not\in \ract{\history[\deltaHist]}(\qDSm)$ holds, because $\tupin$ is already filtered out by $\up'$ (it fulfills $\cond'$). It follows that $\tup \in \dfull$.

%%%%%%%%%%%%%%%%%%%%%%%%%%%%%%%%%%%%%%%%%%%%%%%%%%%%%%%%%%%%%%%%%%%%%%%%%%%%%%%%
\proofpar{\bf Inductive Step:}
We again use $\dfull$ to denote $\iDiff{\ract{\history}(\rel)}{\ract{\history[\deltaHist]}(\rel)}$ and $\dslice$ to denote $\iDiff{\ract{\history}(\qDSh(\rel))}{\ract{\history[\deltaHist]}(\qDSm(\rel))}$ in the following. For a tuple $t$, let $t_i$ be denote $\history_i(t)$ and $t_i'$ to denote ($\ahmod_i(t)$). Abusing notation, let $\modi$ in subscripts of $\tup$ denote the position of a modification $\modi$ in $\history$.

%%%%%%%%%%%%%%%%%%%%%%%%%%%%%%%%%%%%%%%%
\proofpar{$\dfull \subseteq \dslice$:}
% STAGE 1, tuple (successors) have to fulfill some modified update's condition
Let $\tup \in \dfull$. Given that the tuple $\tup \in \dfull$, we will show that this implies that there exists $\modi = \up \gets \up'$ in $\deltaHist$ such that $\cond_\modi$ is the condition of $\up$ and $\cond_{\modi}'$ is the condition of $\up'$ and for which $\cond_\modi(t_{\modi -1}) \vee \cond_{\modi}'(t_{\modi -1}')$. This claim follows from inductive use of the  argument for single modifications from the proof of \Cref{theo:data-slicing-single-mod}. If a tuple $t \in \rel$ is not affected by any of the modified updates (i.e., the successors $t_i$ of $t$ do not fulfill any of the conditions of these statements), then the final result produced by $\history(\tup)$ and $\ahmod(\tup)$ which contradicts $\tup \in \dfull$.

Assume that we have access to an oracle that given an input tuple $\tup$ determines whether any successor of $\tup$ fulfills the condition from above:

\[
  \exists \modi \in \deltaHist: \cond_\modi(t_{\modi -1}) \vee \cond_{\modi}'(t_{\modi -1}')
\]

Then we could filter input tuples using this oracle without changing the result of the historical what-if query. The only problem is that we cannot simply use a selection with condition, because we have only access to $\tup$, but not its successors $t_i$ (the result of applying the first $i$ updates to $t$). In the remainder of the proof we will show how to filter the input based on a pushed down condition is equivalent to applying the condition to a successor. From this then immediately follows the claim.

Since a reenactment query $\ract{\history}$ is equivalent to the history $\history$, it suffices to show that conditions can be pushed through the relational algebra operators used in $\ract{\history}$. We then apply this repetitively for both $\ract{\history}$ and $\ract{\ahmod}$ to yield $\ract{\history}(\qDSh(\rel))$ and $\ract{\ahmod}(\qDSm(\rel))$.
In the following we will abuse notation and for a query $\query$, denote by $\pushCond{\cond}{i}$ the condition $\cond$ pushed through the top-most $i$ operators.
Consider a tuple $\tup = (c_1, \ldots, c_n)$, query $\query$ consisting of a single relational algebra operator, and condition $\cond$ and let $\tup_{out} = \query(\tup)$.
We need to show that
\begin{align}
  \cond(\query(\{\tup\})) = \pushCond{\cond}{1}(\tup)
  \label{eq:push-down-eq}
\end{align}


Showing this equivalency is possible by a case distinction over the possible relational algebra operators:
\begin{itemize}
	\item \textbf{Projection} Consider a projection $\projection_{\vec{A}}$ with $\vec{A} = e_1 \to A_1, \ldots, e_n \to A_n$. Therefore, $\projection_{\vec{A}}(t) = (e_1(c_1), \ldots, e_n(c_n))$. Applying the definition of $\pushCond{\cond}{1}(\tup)$, we get $\pushCond{\cond}{1}(\tup) = (e_1(c_1), \ldots, e_n(c_n)) = \projection_{\vec{A}}(t)$.
	\item \textbf{Selection} Based on the semantics of selection, $\tup_{out} = \tup$. Based on $\pushCond{\cond}{1} = \cond$, it follows that $\pushCond{\cond}{1}(\tup) = \cond(\tup_{out})$.
	\item \textbf{Union} Considering that a union has two inputs, the tuple $\tup$ may be present in the left, the right, or both inputs to the union. However, no matter which input it stems from, $\tup_{out} = \tup$. Since $\pushCond{\cond}{1} = \cond$, we have $\pushCond{\cond}{1}(\tup) = \cond(\tup_{out})$.
\end{itemize}

Applying \Cref{eq:push-down-eq} iteratively, we can push down all conditions to the input table $\rel$ for both reenactment queries $\ract{\history}$ and $\ract{\ahmod}(\rel)$. This condition can then be applied in a selection over $\rel$. Since this is precisely the condition from $\qDSh$ and $\qDSm$, this implies $\dfull \subseteq \dslice$.


%%%%%%%%%%%%%%%%%%%%%%%%%%%%%%%%%%%%%%%%
\proofpar{$\dfull \supseteq \dslice$} There are three cases where $\dslice$ could contain tuples not in $\dfull$. The first two cases are symmetric in form, and they are the cases where a tuple $\tup$ is filtered to exclusively $H$ ($\ahmod$ symmetrically). The third case is when a tuple might be inserted by $\qDSh(\rel)$ ($\qDSm(\rel)$).
%%
We can eliminate the third case using the following argument. First observe that $\qDSh(\rel) \subseteq \rel$ as it applies a selection over the reenactment. Considering our updates are monotone this implies that $\history(\qDSh(\rel)) \subseteq \history(\rel)$. Using the same argument, we also have $\ahmod(\qDSm(\rel)) \subseteq \ahmod(\rel)$. Thus, no new tuples can appear in $\dslice$.
%%
  Thus we now only need to consider the first two cases, in which the symmetric difference could possibly produce tuples outside of $\dfull$. For the sake of contradiction, let $\tup$ be a tuple not in $\dfull$ but present in $\dslice$. For a tuple to be in $\dslice$, it needs to match $\cond_\modi$ or $\cond_\modi'$ for at least one $\modi \in \deltaHist$. However, recall that \Cref{eq:update-ds-cond} takes the disjunction over the conditions for $\history$ and $\ahmod$. That is, if a tuple matches at least one, it is included in the slice of the data, making the case that a tuple is filtered to exclusively one history impossible (any tuple modified by just one of the histories is included for reenactment in the other). It then follows from \Cref{theo:data-slicing-single-mod} that $\tup$ must be in $\dfull$ given that it matches the condition $\cond_\modi$ or $\cond_\modi'$ for at least one $\modi \in \deltaHist$. Therefore, we find that $\forall \tup \in \dslice: \tup \in \dfull$, i.e. $\dfull \supseteq \dslice$.

%%%%%%%%%%%%%%%%%%%%%%%%%%%%%%%%%%%%%%%%
\end{proof}

%%%%%%%%%%%%%%%%%%%%%%%%%%%%%%%%%%%%%%%%%%%%%%%%%%%%%%%%%%%%%%%%%%%%%%%%%%%%%%%%


%%% Local Variables:
%%% mode: latex
%%% TeX-master: "techreport"
%%% End:
}

\iftechreport{
%%%%%%%%%%%%%%%%%%%%%%%%%%%%%%%%%%%%%%%%%%%%%%%%%%%%%%%%%%%%%%%%%%%%%%%%%%%%%%%%
\partitle{Discussion}
%
While for data slicing for historical what-if queries with a single modification, the cost of evaluating the data slicing condition is almost always less then the cost saved by reducing the amount of data to be evaluated by the remainder of the reenactment query. However, for multiple modifications, the cost of conditions for a modification that affects an update later in the input history may approach the cost of the reenactment query itself in the worst-case. This cost depends on several factors: the position of the modified update in the history, the number of attributes referenced by the condition of the update, and how many updates before the modified update have modified attributes referenced by the modified update's condition.
}

%%%%%%%%%%%%%%%%%%%%%%%%%%%%%%%%%%%%%%%%
%\begin{proof}
%We prove it by contradiction. A historical what-if query with a single modification $m = \xid.u \gets u'$, suppose there is a tuple $t \in \db$ and it does not satisfy the condition $\theta(u) \vee \theta(u')$ but it is in the result of the historical what-if query $\qResultDiff{\query}{\ract{\history}}{\ract{\history[\deltaHist]}}$. In this case, it can not be modified by either $u$ or $u'$ since it does not meet the condition of any of these updates. So, $t$ would be same in the result of $\ract{\history}$ and $\ract{\history[\deltaHist]}$ and it can not be in the result of $\qResultDiff{\query}{\ract{\history}}{\ract{\history[\deltaHist]}}$ so we can exclude this tuple and filter the input to $\ract{\history}$ and $\ract{\history[\deltaHist]}$ and this concludes the proof.
%\end{proof}

%%%%%%%%%%%%%%%%%%%%%%%%%%%%%%%%%%%%%%%%
%\begin{theo}
%For a delete statement in the history that is not belong to the modification $ u\notin\deltaHist$, we can filter data that are removed by it from the reenactment query of that delete statement ($\ract{u}$). Therefore, we can use $\selection_{\neg\theta}(\ract{u})$ as the reenactment query of the delete statement.
%$\,$\\[-4mm]
%\end{theo}
%%%%%%%%%%%%%%%%%%%%%%%%%%%%%%%%%%%%%%%%

% We leave the extension of this optimization for inserts with queries (i.e., \lstinline!INSERT INTO ... SELECT!) to future work.
% \BG{This may be confusing since we did not allow SELECT in our definition of inserts. As discussed we probably want to get all restrictions out of the way early on instead of }


%%%%%%%%%%%%%%%%%%%%%%%%%%%%%%%%%%%%%%%%%%%%%%%%%%%%%%%%%%%%%%%%%%%%%%%%%%%%%%%%
\section{Program Slicing}
\label{sec:dep-ana}

% Similar to how we can exclude tuples from reenactment if they are not affected by an update that is changed by the modification of a historical what-if query,

In addition to data slicing,
we also optimize the process of answering a historical what-if query $\hwhatif = (\history, \db, \deltaHist)$
by excluding statements from reenactment if their existence has provably no effect on the answer of $\hwhatif$. This is akin to \textit{program slicing}~\cite{cheney07,W81} which is a technique developed by the PL community to determine a slice (a subset of the statements of a program) that is sufficient for computing the values of variables at a given set of locations in the program.
Analog, we define slices of histories wrt. historical what-if queries.
A slice for a historical what-if query $\hwhatif$ consists of subsets of $\history$ and $\history[\deltaHist]$ that can be substituted for the original history and modified history when evaluating the historical what-if query without changing its result. Recall that the result of a historical what-if query is computed as the delta (symmetric difference) between the result of the original and the modified history. That is, only tuples in the delta are relevant for determining slices.
\revdel{We distinguish between \textit{dynamic slices} which take the database $\db$ into account and \textit{static slices} which are independent of the database and are sufficient for any database.}

\BG{Should we really define this as $\dbver{\idx_1}$ since we actually evaluate over $\db$ even when using the slice}
%%%%%%%%%%%%%%%%%%%%%%%%%%%%%%%%%%%%%%%%
\begin{defi}[History Slices]
  Let $\hwhatif = (\history, \db, \deltaHist)$ be a historical what-if query over a history $\history = (\up_1, \ldots, \up_n)$.
  Furthermore, let $\idxs = \{ \idx_1, \ldots, \idx_m \}$ be a set of indexes from $[1,n]$ such that $\idx_j < \idx_k$ for $j < k$.
  We call  $(\hslice{\history}{\idxs}, \hslice{\history[\deltaHist]}{\idxs})$  a  \emph{slice} for $\hwhatif$ if
  % of \emph{minimal}  $\hslice{\history}$ of  $\history \cup \history[\deltaHist]$ which
  % it fulfills the condition shown below.
  % a \emph{slice} for $\hwhatif$.
\begin{gather*}
  \begin{aligned}
    &\iDiff{\history(\db)}{\history[\deltaHist](\db)} = \iDiff{\hslice{\history}{\idxs}(\dbver{\idx_1})}{\hslice{\history[\deltaHist]}{\idxs}(\dbver{\idx_1})}
  \end{aligned}
\end{gather*}
% In constrast, $(\hslice{\history}{\idxs}, \hslice{\history[\deltaHist]}{\idxs})$ is a \emph{static slice} for $\hwhatif$ if:
% \begin{align*}
% \forall D:  \iDiff{\history(\db)}{\history[\deltaHist](\db)} = \iDiff{\hslice{\history}{\idxs}(\dbver{\idx_1})}{\hslice{\history[\deltaHist]}{\idxs}(\dbver{\idx_1})}
% \end{align*}
\end{defi}
%%%%%%%%%%%%%%%%%%%%%%%%%%%%%%%%%%%%%%%%

History slices allow us to optimize the evaluation of a historical what-if query by excluding statements from reenactment. Thus, ideally, we would like slices  to be \emph{minimal}, i.e., the result of removing any statement from $\hslice{\history}{\idxs}$ or $\hslice{\history[\deltaHist]}{\idxs}$ is not a slice.
\iftechreport{There may exist more than one minimal slice for a query $\hwhatif$, because the exclusion of one statement may prevent us from excluding another statement.}
\revdel{Note that the condition for static slices is more strict than the condition for dynamic slices, because what subsets of the histories are minimal dynamic slices for a historical what-if query may depend on $\db$. Thus, any static slice is a dynamic slice, but not vice versa and a minimal dynamic slice for a query $\hwhatif$ may be smaller than any static slice for $\hwhatif$.}
A naïve method for testing whether $\idxs$ is a slice, is to compute $\iDiff{\history(\db)}{\history[\deltaHist](\db)}$ and compare it against $\iDiff{\hslice{\history}{\idxs}(\dbver{\idx_1})}{\hslice{\history[\deltaHist]}{\idxs}(\dbver{\idx_1})}$. However, this is more expensive then just directly evaluating $\iDiff{\history(\db)}{\history[\deltaHist](\db)}$ which we wanted to optimize. Instead we give up minimality and restrict program slicing to tuple independent statements (\Cref{def:tuple-independence}) which enables us to check that the slice and full histories produce the same result one tuple at a time. Furthermore, we
  design a method that (lossily) compresses the database $\db_C$ and checks this condition (same result for each input tuple) over the compressed database. Since the compression is lossy, a compressed database $\db_C$ represents all databases $\db$ such that compressing $\db$ yields $\db_C$. To ensure that our method produces a slice that is valid for each such $\db$, we adapt techniques from incomplete databases~\cite{IL84a, lenses15}.
\revdel{Based on this discussion, the reader may assume that dynamic slices are preferable. However, as we have argued above, we have to access the database to compute a minimal dynamic slice while, as we will demonstrate in the following, static slices can be computed based on $\history$ and $\ahmod$ alone. That is, the cost of computing static slices only depends on the length of the history, number of attributes in the schema of the relation updated by $\history$, and the size of expressions in the statements of the history, but is independent of the size of the input database.
% However, evaluating the histories over $\db$ to determine a minimal slice (if this is even computationally feasible) defies the purpose of program slicing, because it require evaluating the operation we wanted to optimize.
Thus, we focus on developing a method that computes static slices independent of $\db$ which based on the above observation implies that the slices determined by our method can not be minimal in general.}
% to reduce the amount of work required to evaluate a historical what-if query.

% defies the purpose of program slicing

% is likely to be more costly than just evaluating the unsliced histories.

% \FC{Where are these fragments supposed to go?}


%%% Local Variables:
%%% mode: latex
%%% TeX-master: "historical_whatif"
%%% End:
