\newcommand{\naivedbmod}{\db_{mod}}
\newcommand{\naivedbcur}{\db_{cur}}

%%%%%%%%%%%%%%%%%%%%%%%%%%%%%%%%%%%%%%%%
\begin{algorithm}[t]
  \caption{Naïve \abbrHW Algorithm}
  \label{alg:naive-sol}
  \begin{algorithmic}[1]
    \Procedure{Naive-WhatIf}{$\history$, $\db$, $\naivedbcur$, $\deltaHist$}
    \State $\db' \gets \Call{Copy}{\db}$ \label{alg-line:naive-dbzero}
    \State $\naivedbmod \gets \history[\deltaHist](\db')$ \label{alg-line:naive-mod-history}
    \State \Return $\iDiff{\naivedbcur}{\naivedbmod}$ \label{alg-line:naive-sym-diff}
    \EndProcedure
  \end{algorithmic}
\end{algorithm}
%%%%%%%%%%%%%%%%%%%%%%%%%%%%%%%%%%%%%%%%

%%%%%%%%%%%%%%%%%%%%%%%%%%%%%%%%%%%%%%%%%%%%%%%%%%%%%%%%%%%%%%%%%%%%%%%%%%%%%%%%
\section{Naïve Algorithm}\label{sec:naive-solution}

Before giving an overview of our approach, we briefly revisit the naïve algorithm (\Cref{alg:naive-sol}) in more detail. WLOG assume that $\deltaHist$ modifies the first update in the history (and possibly others). If this is not the case, then we can simply ignore the prefix of the history before the first modified statement and use the state of the database before that statement instead of the database before first statement in the history. The input to the algorithm is the history $\history$, the database state before the first statement of $\history$ was executed ($\db)$, the current state of the database $\naivedbcur$ which is assumed to be equal to $\history(\db)$, and the modifications $\deltaHist$ of the historical what-if  query $\hwhatif$. We assume that $\db$ can be accessed using time travel.
% 
The algorithm first creates a copy of $\db'$ of $\db$.  % the database state before the first statement of $\history$.
% 
Note that we only need to copy relations that are accessed by the history. The state of any relation not accessed by $\history$ will be the same in $\history(\db)$ and $\ahmod(\db')$.  We rename the relations in $\db'$ to avoid name clashes.
We then execute $\ahmod$ over the copy $\db'$  resulting in $\naivedbmod = \ahmod(\db')$ (\Cref{alg-line:naive-mod-history}).
In the last step (\Cref{alg-line:naive-mod-history}), the delta of $\naivedbcur$ and $\naivedbmod$ is computed. The delta computation is implemented as a single query for each relation of $\db$ accessed by $\history$. For instance, a relational algebra query computing the delta for a  relation $\rel$ with schema $\schema{\rel} = (A,B)$ is shown below. Note that $+$ and $-$ are constants, i.e., the projections add an additional column storing the annotation of a tuple.

\[
  \projection_{A,B,-}(\rel_{cur} - \rel_{mod}) \union   \projection_{A,B,+}(\rel_{mod} - \rel_{cur})
\]


%%%%%%%%%%%%%%%%%%%%%%%%%%%%%%%%%%%%%%%%
\begin{algorithm}[t]
  \caption{Optimized, Reenactment-based \abbrHW Algorithm}
  \label{alg:whatif-algo}
  \begin{algorithmic}[1]
    \Procedure{WhatIf}{$\history$, $\db$, $\deltaHist$}
    \State $\idxs \gets \Call{ProgramSlicing}{\history,\ahmod}$ \Comment{Compute Slice $\idxs$} \label{alg-line:opt-ps}
    \State $\ract{\hslice{\history}{\idxs}} \gets \Call{GenReenactmentQuery}{\hslice{\history}{\idxs}}$ \label{alg-line:opt-h-renact}
    \State $\ract{\hslice{\history}{\idxs}}^{DS} \gets \Call{DataSlicing}{\history,\deltaHist,\ract{\hslice{\history}{\idxs}}}$ \label{alg-line:opt-h-ds}
    \State $\ract{\hslice{\ahmod}{\idxs}} \gets \Call{GenReenactmentQuery}{\hslice{\ahmod}{\idxs}}$ \label{alg-line:opt-m-renact}
    \State $\ract{\hslice{\ahmod}{\idxs}}^{DS} \gets \Call{DataSlicing}{\history,\deltaHist,\hslice{\ahmod}{\idxs}}$ \label{alg-line:opt-m-ds}
    \State \Return $\iDiff{\ract{\hslice{\history}{\idxs}}^{DS}}{\ract{\hslice{\ahmod}{\idxs}}^{DS}}$ \label{alg-line:opt-diff}
    \EndProcedure
  \end{algorithmic}
\end{algorithm}
%%%%%%%%%%%%%%%%%%%%%%%%%%%%%%%%%%%%%%%%

%%%%%%%%%%%%%%%%%%%%%%%%%%%%%%%%%%%%%%%%%%%%%%%%%%%%%%%%%%%%%%%%%%%%%%%%%%%%%%%%
\section{Overview of Our Approach}
\label{sec:overview}

% Answering historical what-if queries requires us to have access to a log of SQL commands executed in the past (audit log) and query access to the past states of tables (time travel). For databases that support these features (e.g., Oracle, DB2, and MSSQL), we can use an audit log that stores a history of SQL commands executed in the past and for each command when it was executed and as part of which transaction. Otherwise, we can use triggers to capture SQL commands or exploit other logging mechanisms if available.

We now give a high-level overview of our \Cref{alg:whatif-algo} for answering a \abbrHW $\hwhatif = (\history, \db, \deltaHist)$.  % that we already introduce informally in \Cref{sec:introduction}.
To answer a historical what-if query, we need to compute $\history(\db)$ and $\ahmod(\db)$, and compute the delta of $\history(\db)$ and $\ahmod(\db)$. As mentioned earlier, we utilize a technique called reenactment for this purpose. In the following we first give an overview of reenactment and then discuss how it is applied by our approach. % and briefly discuss optimization techniques that we have developed for answering historical what-if queries.


%%%%%%%%%%%%%%%%%%%%%%%%%%%%%%%%%%%%%%%%%%%%%%%%%%%%%%%%%%%%%%%%%%%%%%%%%%%%%%%%
\subsection{Reenactment}
\label{sec:reenactment}

Reenactment~\cite{AG18,AG14} is a technique
for simulating a transactional history through queries. For simplicity we limit the discussion to a history $\history$ over a single relation $\rel$ even though our approach supports histories over multiple relations. Using reenactment, we
can construct a query $\ract{\history}$ such that
$\history(\rel) = \ract{\history}(\rel)$\iftechreport{\footnote{\cite{AG18} did prove a
  stronger result, demonstrating equivalence for annotated relations which
  implies equivalence for set and bag semantics as a special case.}}. Reenactment
was originally developed for capturing provenance for transactional workloads
under multiversioning concurrency control protocols. For our purpose, we only
need reenactment for set semantics and introduce a simplified translation for this case.
We use $\ract{\up}$ ($\ract{\history}$) to denote the reenactment query for a single statement $\up$ (history $\history$).

%%%%%%%%%%%%%%%%%%%%%%%%%%%%%%%%%%%%%%%%%%%%%%%%%%%%%%%%%%%%%%%%%%%%%%%%%%%%%%%%
\begin{defi}[Reenactment Queries]\label{def:reenactment-queries}
  Let be a statement $\up$ (update $\aupdate$, delete $\adelete$, insert $\ainsert$, or insert $\aqinsert$) over a relation $\rel$ with schema $(A_1, \ldots, A_n)$ and let $\pset = (\expr_1, \ldots, \expr_n)$. The reenactment query  $\ract{\up}$ for $\up$ is defined as shown below:
%
  \begin{align*}
    \ract{\aupdate} & \defas \projection_{\sqlCase{\cond}{e_1}{A_1}, \ldots, \sqlCase{\cond}{e_n}{A_n}}(\rel)
  \end{align*}\\[-9mm]
  %%%%%%%%%%%%%%%%%%%%%%%%%%%%%%%%%%%%%%%%
  \begin{align*}
        \ract{\adelete} & \defas \selection_{\neg\, \cond}(\rel) &
    \ract{\ainsert} & \defas \rel \union \{ t \} &
    \ract{\aqinsert} & \defas \rel \union \query
  \end{align*}
  %%%%%%%%%%%%%%%%%%%%%%%%%%%%%%%%%%%%%%%%

  %   \ract{\adelete} & \defas \selection_{\neg\, \cond}(\rel) \\
  %   \ract{\ainsert} & \defas \rel \union \{ t \}\\
  %   \ract{\aqinsert} & \defas \rel \union \query
  % \end{align*}
%
Let $\history = (\up_1, \ldots, \up_n)$ be a history. The reenactment query $\ract{\history}$ for $\history$ is constructed from the reenactment queries for $\up_i$ for $i \in \{1,\ldots,n\}$ by substituting the reference to relation $\rel$ in $\ract{\up_i}$ with $\ract{u_{i-1}}$.
\end{defi}
%%%%%%%%%%%%%%%%%%%%%%%%%%%%%%%%%%%%%%%%%%%%%%%%%%%%%%%%%%%%%%%%%%%%%%%%%%%%%%%%

An insert is reenacted as the union between the current state of relation $R$ and the inserted tuple ($\ainsert$) or the result of query $\query$ (for $\aqinsert$). For a delete $\adelete$, we have to remove all tuples fulfilling the condition of the delete. This is achieved by using a selection to only retain tuples that do not fulfill this condition, i.e., we filter based on $\neg \cond$. To reenact an update, we have to update the attribute values of all tuples fulfilling the condition $\cond$ using the expressions $\pset$. All other tuples are just copied from the input. For that, we project on conditional expressions that for each attribute $A_i$ return $\expr_i$ if the tuple fulfills $\cond$ and $A_i$ otherwise.
For a history $\history$ which accesses multiple relations, a separate query,  $\subract{\rel}{\history}$, is constructed for each relation $\rel$ based on all statements from  history $\history$ that access $\rel$.

%%%%%%%%%%%%%%%%%%%%%%%%%%%%%%%%%%%%%%%%
\begin{exam}\label{ex:reenact-example}
%In this example, we examine the transitive dependency of transactions.
  Consider \Cref{ex:running-example} and let $I,U,C,P$, and $F$ denote attributes \emph{ID}, \emph{Customer}, \emph{Country}, \emph{Price}, and \emph{ShippingFee} of relation \emph{Order} (abbreviated as \emph{O}). The reenactment query $\subract{O}{\history}$ for the history $\history$ from \Cref{fig:Transitive-Transactions-Example} is:

\vspace{-1mm}
\resizebox{1\linewidth}{!}{
\begin{minipage}{1.0\linewidth}
\begin{align*}
  \subract{O}{\history} & = \projection_{I,U,C,P,\sqlCase{P \leq 30 \land F \geq 10}{F-2}{F}}(                                                                                                                                                                                              \\
                        &\hspace{4mm}\projection_{I,U,C,P,\sqlCase{U = UK \land P \leq 100}{F+5}{F}}(                                                                                                                                                                                                   \\
                        &\hspace{4mm} \projection_{I,U,C,P,\sqlCase{P \geq 50}{0}{F}}(O)))\\[-3mm]
                      % \ract^{O}(u_3)                                                                                                                                                                                                                                                         \\
% \subract{O}{u_3}      & = \projection_{ID,cu,co,p,(sf-2) \to sf}(\selection_{ p<=30 \wedge sf>=10}(\ract^O(\up_2)))                                                                                                                                                                          \\
%                       & \mathtab \union \selection_{\neg ( p<=30 \wedge sf>=10)}(\ract^O(\up_2))                                                                                                                                                                                             \\
% \ract^{O}(u_2)        & = \projection_{ID,cu,co,p,(sf+5) \to sf}(\selection_{co=\text{\upshape \textquotesingle}UK\text{\upshape \textquotesingle} \wedge p<=100}(\ract^O(\up_1)))                                                                                                           \\
%                       & \mathtab \union \selection_{\neg (co=\text{\upshape \textquotesingle}UK\text{\upshape \textquotesingle} \wedge p<=100)}(\ract^O(\up_1))                                                                                                                              \\
% \ract^{O}(u_1)        & = \projection_{ID,cu,co,p,0 \to sf}(\selection_{p>=40}(\relCV{O}{20}))                                                                                                                                                                                               \\
%                       & \mathtab \union \selection_{\neg (p>=40)}(\relCV{O}{20})                                                                                                                                                                                                             \\
\end{align*}
\end{minipage}
}

Recall that $\history[\deltaHist]$ differs from $\history$ in that ${\up_1}'$ replaces $\up_1$ and that the condition of ${\up_1}'$ is $P \geq 60$. Thus,
$\subract{O}{\history[\deltaHist]}$ differs from $\subract{O}{\history}$ in that condition $P \geq 50$ in the first selection is replaced with $P \geq 60$.
\end{exam}
%%%%%%%%%%%%%%%%%%%%%%%%%%%%%%%%%%%%%%%%







%%%%%%%%%%%%%%%%%%%%%%%%%%%%%%%%%%%%%%%%%%%%%%%%%%%%%%%%%%%%%%%%%%%%%%%%%%%%%%%%
\subsection{Reenacting Historical What-if Queries}
\label{sec:answ-hist-what}

As shown above, we use reenactment to simulate the evaluation of histories. Given the reenactment queries for $\history$ and $\ahmod$, what remains to be done is to compute their delta.
%(t=\text{\upshape \textquotesingle}Savings\text{\upshape \textquotesingle})
Continuing with our example from above, the result $\iDiff{\subract{O}{\history}(\db)}{\subract{O}{\ahmod}(\db)}$ of $\hwhatif$ is computed as shown below.

\begin{align*}
  \iDiff{\subract{O}{\history}(\db)}{\subract{O}{\history[\deltaHist]}(\db)}&=  \projection_{I,U,C,P,F,-}(\subract{O}{\history}(\db)- \subract{O}{\history[\deltaHist]}(\db))\\
                                                                            &\hspace{4mm}  \union \projection_{I,U,C,P,F,+}(\subract{O}{\history[\deltaHist]}(\db)-\subract{O}{\history}(\db))
\end{align*}

\BG{Possibly shorten:}
We use \Cref{alg:whatif-algo} to answer historical what-if queries. This algorithm applies two novel optimizations that significantly improve  performance. Program slicing (\Cref{alg-line:opt-ps}, discussed in \Cref{sec:dep-ana})  determines subsets of histories (encoded as a set of positions $\idxs$ called a \emph{slice}) which are sufficient for computing the answer to the what-if query $\hwhatif$. We then generate reenactment queries (\Cref{alg-line:opt-h-renact,alg-line:opt-m-renact})  for the slices of $\history$ and $\ahmod$ according to $\idxs$. Recall that $\hslice{\history}{\idxs}$ denotes the history generated from $\history$ by removing all statements not in $\idxs$. Afterwards (\Cref{alg-line:opt-h-ds,alg-line:opt-m-ds}), we apply our second optimization, data slicing (discussed in \Cref{sec:filter}). Data slicing injects selection conditions into the reenactment query that filter out data that is irrelevant for computing the result of the \abbrHW. The result of data and program slicing is an optimized version of a reenactment query that has to process significantly less data and avoids reenacting updates that are irrelevant for $\hwhatif$. We then calculate the delta of these two queries and return it as the answer for $\hwhatif$ (\Cref{alg-line:opt-diff}).


%%%%%%%%%%%%%%%%%%%%%%%%%%%%%%%%%%%%%%%%


% %%%%%%%%%%%%%%%%%%%%%%%%%%%%%%%%%%%%%%%%%%%%%%%%%%%%%%%%%%%%%%%%%%%%%%%%%%%%%%%%
% \subsection{Optimized Reenactment}
% \label{sec:overv-optim}



%%% Local Variables:
%%% mode: latex
%%% TeX-master: "historical_whatif"
%%% End:
